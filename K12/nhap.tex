\chapter{KHÁI NIỆM VECTƠ}

% =======================
% PHẦN I. KIẾN THỨC CƠ BẢN
% =======================
\section{Kiến thức cơ bản}

I. KHÁI NIỆM VECTƠ
Cho đoạn thẳng AB. Nếu chọn điểm A làm điểm đầu, điểm B làm điểm cuối thì đoạn thẳng AB có hướng từ A đến B. Khi đó ta nói AB là một đoạn thẳng có hướng.

\subsection{Định nghĩa}
\begin{dn}
    Vectơ là một đoạn thẳng có hướng, nghĩa là, trong hai điểm mút của đoạn thẳng, đã chỉ rõ điểm đầu, điểm cuối.
    \begin{center}
        A \quad \rule{2cm}{0.5pt} \quad B
    \end{center}
    Vectơ có điểm đầu A và điểm cuối B được kí hiệu là $\vec{AB}$, đọc là “vectơ AB”.
    Vectơ còn được kí hiệu là $\vec{a}, \vec{b}, \vec{x}, \vec{y}, \dots$ khi không cần chỉ rõ điểm đầu và điểm cuối của nó.
    Độ dài của vectơ là khoảng cách giữa điểm đầu và điểm cuối của vectơ đó.
\end{dn}

\subsection{Chú ý}
\begin{note}
    Độ dài của vectơ $\vec{AB}$ được kí hiệu là $|\vec{AB}|$, như vậy $|\vec{AB}| = AB$. Độ dài của vectơ $\vec{a}$ được kí hiệu là $|\vec{a}|$.
\end{note}
Vectơ có độ dài bằng 1 gọi là vectơ đơn vị.

II. HAI VECTƠ CÙNG PHƯƠNG, CÙNG HƯỚNG
\subsection{Định nghĩa}
\begin{dn}
    Giá của vectơ: Đường thẳng đi qua điểm đầu và điểm cuối của một vectơ được gọi là giá của vectơ đó.

    Vectơ cùng phương, vectơ cùng hướng: Hai vectơ được gọi là cùng phương nếu giá của chúng song song hoặc trùng nhau.
    Hai vectơ cùng phương thì chúng chỉ có thể cùng hướng hoặc ngược hướng.
\end{dn}

\ghinho{Ba điểm phân biệt A, B, C thẳng hàng khi và chỉ khi hai vectơ $\vec{AB}$ và $\vec{AC}$ cùng phương.}

III. HAI VECTO BẰNG NHAU:
\subsection{Định nghĩa}
\begin{dn}
    Hai vectơ $\vec{a}$ và $\vec{b}$ được gọi là bằng nhau nếu chúng cùng hướng và có cùng độ dài.
    Kí hiệu $\vec{a} = \vec{b}$.
\end{dn}

\subsection{Chú ý}
\begin{note}
    Hai vectơ $\vec{a}$ và $\vec{b}$ được gọi là đối nhau nếu chúng ngược hướng và có cùng độ dài.
    Khi cho trước vectơ $\vec{a}$ và điểm O, thì ta luôn tìm được một điểm A duy nhất sao cho $\vec{OA} = \vec{a}$.
\end{note}

IV. VECTƠ – KHÔNG
\subsection{Định nghĩa}
\begin{dn}
    Vectơ – không là vectơ có điểm đầu và điểm cuối trùng nhau, ta kí hiệu là $\vec{0}$.
    Ta quy ước vectơ – không cùng phương, cùng hướng với mọi vectơ và có độ dài bằng 0.
    Như vậy $\vec{0} = \vec{AA} = \vec{BB} = \dots$ và $\vec{MN} = \vec{0} \Leftrightarrow M = N$.
\end{dn}

% =======================
% PHẦN II. CÁC DẠNG BÀI TẬP
% =======================
\section{Các dạng bài tập}

\begin{dang}{XÁC ĐỊNH MỘT VECTƠ; PHƯƠNG, HƯỚNG CỦA VECTƠ; ĐỘ DÀI CỦA VECTO}
    \begin{pp}
        \item Xác định một vectơ và xác định sự cùng phương, cùng hướng của hai vectơ theo định nghĩa.
        \item Dựa vào các tính chất hình học của các hình đã cho biết để tính độ dài của một vectơ.
    \end{pp}

    \begin{vd}
        Với hai điểm phân biệt A, B có thể xác định được bao nhiêu vectơ khác vectơ-không có điểm đầu và điểm cuối được lấy từ hai điểm trên?
        \loigiaiEX{
            Hai vectơ $\vec{AB}$ và $\vec{BA}$.
        }
    \end{vd}

    \begin{vd}
        Cho tam giác ABC, có thể xác định được bao nhiêu vectơ khác vectơ-không có điểm đầu và điểm cuối là các đỉnh A, B, C?
        \noteImage[width=0.45\linewidth]{data//im/c3b4-v2.png}
        \loigiaiEX{
            Ta có 6 vectơ: $\vec{AB}, \vec{BA}, \vec{BC}, \vec{CB}, \vec{CA}, \vec{AC}$.
        }
    \end{vd}

    \begin{vd}
        Cho hình lục giác đều ABCDEF tâm O. Tìm số các vectơ khác vectơ - không, cùng phương với vectơ $\vec{OB}$ có điểm đầu và điểm cuối là các đỉnh của lục giác?
        \noteImage[width=0.45\linewidth]{data//im/c3b4-v3.png}
        \loigiaiEX{
            Các vectơ cùng phương với vectơ $\vec{OB}$ là:
            $\vec{BE}, \vec{EB}, \vec{DC}, \vec{CD}, \vec{FA}, \vec{AF}$.
        }
    \end{vd}

    \begin{vd}
        Cho lục giác đều ABCDEF tâm O. Tìm số các vectơ bằng $\vec{OC}$ có điểm đầu và điểm cuối là các đỉnh của lục giác?
        \noteImage[width=0.45\linewidth]{data//im/c3b4-v4.png}
        \loigiaiEX{
            Đó là các vectơ: $\vec{AB}, \vec{ED}$.
        }
    \end{vd}

    \begin{vd}
        Cho hình bình hành ABCD tâm O. Gọi P, Q, R lần lượt là trung điểm của AB, BC, AD. Lấy 8 điểm trên là gốc hoặc ngọn của các vectơ. Tìm số vectơ bằng với vecto $\vec{AR}$
        \noteImage[width=0.45\linewidth]{data//im/c3b4-v5.png}
        \loigiaiEX{
            Có 3 vectơ là $\vec{RD}; \vec{BQ}; \vec{QC}, \vec{PO}$.
        }
    \end{vd}

    \begin{vd}
        Cho tứ giác ABCD. Có bao nhiêu vectơ khác vectơ không có điểm đầu và cuối là các đỉnh của tứ giác?
        \loigiaiEX{
            Một vectơ khác vectơ không được xác định bởi 2 điểm phân biệt. Khi có 4 điểm A, B, C, D ta có 4 cách chọn điểm đầu và 3 cách chọn điểm cuối. Nên ta sẽ có $3 \cdot 4 = 12$ cách xác định số vectơ khác $\vec{0}$ thuộc 4 điểm trên.
        }
    \end{vd}

    \begin{vd}
        Số vectơ (khác vectơ $\vec{0}$) có điểm đầu và điểm cuối lấy từ 7 điểm phân biệt cho trước?
        \loigiaiEX{
            Một vectơ khác vectơ không được xác định bởi 2 điểm phân biệt. Khi có 7 điểm ta có 7 cách chọn điểm đầu và 6 cách chọn điểm cuối. Nên ta sẽ có $7 \cdot 6 = 42$ cách xác định số vectơ khác $\vec{0}$ thuộc 7 điểm trên.
        }
    \end{vd}

    \begin{vd}
        Trên mặt phẳng cho 6 điểm phân biệt A,B,C,D,E; F . Hỏi có bao nhiêu vectơ khác vecto - không, mà có điểm đầu và điểm cuối là các điểm đã cho?
        \loigiaiEX{
            Xét tập $X = \{A, B, C, D, E ; F\}$. Với mỗi cách chọn hai phần tử của tập $X$ và sắp xếp theo một thứ tự ta được một vectơ thỏa mãn yêu cầu.
            Mỗi vectơ thỏa mãn yêu cầu tương ứng cho ta $30$ phần tử thuộc tập $X$.
            Vậy số các vectơ thỏa mãn yêu cầu bằng $30$.
        }
    \end{vd}

    \begin{vd}
        Cho $n$ điểm phân biệt. Hãy xác định số vectơ khác vectơ $\vec{0}$ có điểm đầu và điểm cuối thuộc $n$ điểm trên?
        \loigiaiEX{
            Khi có $n$ điểm, ta có $n$ cách chọn điểm đầu và $n-1$ cách chọn điểm cuối. Nên ta sẽ có $n(n-1)$ cách xác định số vectơ khác $\vec{0}$ thuộc $n$ điểm trên.
        }
    \end{vd}

    \begin{vd}
        Cho lục giác đều ABCDEF tâm O. Số các vectơ bằng $\vec{OC}$ có điểm cuối là các đỉnh của lục giác là bao nhiêu?
        \noteImage[width=0.45\linewidth]{data//im/c3b4-v10.png}
        \loigiaiEX{
            Đó là các vectơ: $\vec{AB}; \vec{ED}$.
        }
    \end{vd}

    \begin{vd}
        Cho ba điểm M, N, P thẳng hàng, trong đó điểm N nằm giữa hai điểm M và P. Tìm các cặp vectơ cùng hướng?
        \noteImage[width=0.45\linewidth]{data//im/c3b4-v11.png}
        \loigiaiEX{
            Các vec tơ cùng hướng là : $\vec{MN}$ và $\vec{MP}$, $\vec{MN}$ và $\vec{NP}$, $\vec{PM}$ và $\vec{PN}$, $\vec{PN}$ và $\vec{NM}$.
        }
    \end{vd}

    \begin{vd}
        Cho hình bình hành ABCD. Tìm vectơ khác $\vec{0}$, cùng phương với vectơ $\vec{AB}$ và có điểm đầu, điểm cuối là đỉnh của hình bình hành ABCD.
        \noteImage[width=0.45\linewidth]{data//im/c3b4-v12.png}
        \loigiaiEX{
            Các vectơ cùng phương với $\vec{AB}$ mà thỏa mãn điều kiện đầu Câu là: $\vec{BA}, \vec{CD}, \vec{DC}$.
        }
    \end{vd}

    \begin{vd}
        Cho lục giác đều ABCDEF tâm O. Tìm số các vectơ khác vectơ không, cùng phương với $\vec{OC}$ có điểm đầu và điểm cuối là các đỉnh của lục giác là:
        \noteImage[width=0.45\linewidth]{data//im/c3b4-v13.png}
        \loigiaiEX{
            Đó là các vectơ: $\vec{AB}, \vec{BA}, \vec{DE}, \vec{ED}, \vec{FC}, \vec{CF}, \vec{OF}, \vec{FO}$.
        }
    \end{vd}

    \begin{vd}
        Cho điểm A và véctơ $\vec{a}$ khác $\vec{0}$. Tìm điểm M sao cho:
        a) $\vec{AM}$ cùng phương với $\vec{a}$.
        b) $\vec{AM}$ cùng hướng với $\vec{a}$.
        \noteImage[width=0.45\linewidth]{data//im/c3b4-v14.png}
        \loigiaiEX{
            Gọi $\Delta$ là giá của $\vec{a}$.
            a) Nếu $\vec{AM}$ cùng phương với $\vec{a}$ thì đường thẳng AM song song với $\Delta$. Do đó M thuộc đường thẳng $m$ đi qua A và song song với $\Delta$. Ngược lại, mọi điểm M thuộc đường thẳng $m$ thì $\vec{AM}$ cùng phương với $\vec{a}$. Chú ý rằng nếu A thuộc đường thẳng $\Delta$ thì $m$ trùng với $\Delta$.
            b) Lập luận tương tự như trên, ta thấy các điểm M thuộc một nửa đường thẳng gốc A của đường thẳng $m$. Cụ thể, đó là nửa đường thẳng chưa điểm E sao cho $\vec{AE}$ và $\vec{a}$ cùng hướng.
        }
    \end{vd}

    \begin{vd}
        Cho tam giác ABC có trực tâm H. Gọi D là điểm đối xứng với B qua tâm O của đường tròn ngoại tiếp tam giác ABC. Chứng minh rằng $\vec{HA} = \vec{CD}$ và $\vec{AD} = \vec{HC}$.
        \noteImage[width=0.45\linewidth]{data//im/c3b4-v15.png}
        \loigiaiEX{
            Ta có $AH \perp BC$ và $DC \perp BC$ (do góc $DCB$ chắn nửa đường tròn). Suy ra $AH || DC$.
            Tương tự ta cũng có $CH || AD$.
            Suy ra tứ giác ADCH là hình bình hành. Do đó $\vec{HA} = \vec{CD}$ và $\vec{AD} = \vec{HC}$.
        }
    \end{vd}

    \begin{vd}
        Cho tam giác ABC vuông cân tại A, có $AB = AC = 4$. Tính $|\vec{BC}|$
        \loigiaiEX{
            Vì $|\vec{BC}| = BC = \sqrt{AB^2 + AC^2} = \sqrt{16+ 16} = 4\sqrt{2}$.
        }
    \end{vd}

    \begin{vd}
        Cho hình vuông ABCD có độ dài cạnh 3. Giá trị của $|\vec{AC}|$ là bao nhiêu?
        \noteImage[width=0.45\linewidth]{data//im/c3b4-v17.png}
        \loigiaiEX{
            Vì $|\vec{AC}| = AC = 3\sqrt{2}$.
        }
    \end{vd}

    \begin{vd}
        Cho tam giác đều ABC cạnh a. Tính $|\vec{CB}|$
        \loigiaiEX{
            Vì $|\vec{CB}| = CB = a$.
        }
    \end{vd}

    \begin{vd}
        Gọi G là trọng tâm tam giác vuông ABC với cạnh huyền BC = 12. Tính $|\vec{GM}|$ (với M là trung điểm của BC)
        \loigiaiEX{
            Vì $|\vec{GM}| = GM = \frac{1}{3}AM = \frac{1}{3} \cdot \frac{1}{2}BC = \frac{1}{3} \cdot \frac{1}{2} \cdot 12 = 2$.
        }
    \end{vd}

    \begin{vd}
        Cho hình chữ nhật ABCD, có $AB = 4$ và $AC = 5$. Tìm độ dài vectơ $\vec{AC}$.
        \loigiaiEX{
            Vì $|\vec{AC}| = AC = 5$.
        }
    \end{vd}
\end{dang}

\begin{dang}{CHỨNG MINH HAI VECTƠ BẰNG NHAU}
    \begin{pp}
        \item Để chứng minh hai vectơ bằng nhau ta chứng minh chúng có cùng độ dài và cùng hướng hoặc dựa vào nhận xét nếu tứ giác ABCD là hình bình hành thì $\vec{AB} = \vec{DC}$ hoặc $\vec{AD} = \vec{BC}$.
    \end{pp}

    \begin{vd}
        Cho hình vuông ABCD tâm O. Hãy liệt kê tất cả các vectơ bằng nhau nhận đỉnh và tâm của hình vuông làm điểm đầu và điểm cuối.
        \noteImage[width=0.45\linewidth]{data//im/c3b4-v21.png}
        \loigiaiEX{
            Các vectơ bằng nhau nhận đỉnh và tâm của hình vuông làm điểm đầu và điểm cuối là:
            $\vec{AB} = \vec{DC}, \vec{AD} = \vec{BC}, \vec{BA} = \vec{CD}, \vec{DA} = \vec{CB}, \vec{AO} = \vec{OC}, \vec{OA} = \vec{CO}, \vec{BO} = \vec{OD}, \vec{OB} = \vec{DO}$.
        }
    \end{vd}

    \begin{vd}
        Cho vectơ $\vec{AB}$ và một điểm C. Có bao nhiêu điểm D thỏa mãn $\vec{AB} = \vec{CD}$.
        \loigiaiEX{
            Nếu C nằm trên đường thẳng AB thì D cũng nằm trên đường thẳng AB.
            Nếu C không nằm trên đường thẳng AB thì tứ giác ABDC là hình bình hành. Khi đó D nằm trên đường thẳng đi qua C và song song với đường thẳng AB .
            Do vậy, có vô số điểm D thỏa mãn $\vec{AB} = \vec{CD}$.
        }
    \end{vd}

    \begin{vd}
        Cho tứ giác đều ABCD. Gọi M,N,P,Q lần lượt là trung điểm của AB, BC, CD, DA. Chứng minh $\vec{MN} = \vec{QP}$.
        \noteImage[width=0.45\linewidth]{data//im/c3b4-v23.png}
        \loigiaiEX{
            Ta có
            \[
                \left\{
                \begin{array}{l}
                    MN//AC \\
                    PQ//AC
                \end{array}
                \right.
                \Rightarrow \left\{
                \begin{array}{l}
                    \text{$\vec{MN}$ cùng phương với $\vec{AC}$} \\
                    \text{$\vec{PQ}$ cùng phương với $\vec{AC}$}
                \end{array}
                \right.
                \Rightarrow MN || PQ \Rightarrow \vec{MN} = \vec{QP}.
            \]
            Hoặc:
            Ta có MN là đường trung bình tam giác $ABC \Rightarrow \vec{MN} \parallel \frac{1}{2}\vec{AC}$ và PQ là đường trung bình
            tam giác $DAC \Rightarrow \vec{PQ} \parallel \frac{1}{2}\vec{AC}$. Do đó $\vec{MN} \parallel \vec{PQ} \Rightarrow MNPQ$ là hình bình hành nên suy ra
            $\vec{MN} = \vec{QP}; \vec{NP} = \vec{MQ}$.
        }
    \end{vd}

    \begin{vd}
        Cho tứ giác ABCD. Điều kiện nào là điều kiện cần và đủ để $\vec{AB} = \vec{CD}$?
        \loigiaiEX{
            Ta có:
            $\vec{AB}=\vec{CD} \Rightarrow \left\{
            \begin{array}{l}
                AB || CD \\
                AB = CD
            \end{array}
            \right. \Rightarrow ABDC$ là hình bình hành.

            Mặt khác, ABDC là hình bình hành $\Rightarrow \left\{
            \begin{array}{l}
                AB || CD \\
                AB = CD
            \end{array}
            \right. \Rightarrow \vec{AB}=\vec{CD}$.

            Do đó, điều kiện cần và đủ để $\vec{AB} = \vec{CD}$ là ABDC là hình bình hành.
        }
    \end{vd}

    \begin{vd}
        Cho hai điểm phân biệt A, B . Xác định điều kiện để điểm I là trung điểm AB .
        \loigiaiEX{
            Vì I là trung điểm AB nên ta có $\vec{IA}+\vec{IB}=\vec{0} \Leftrightarrow \vec{IA}=-\vec{IB} \Leftrightarrow \vec{IA} = \vec{BI}$.
            Vậy điều kiện để điểm I là trung điểm AB là: $\vec{IA} = \vec{BI}$.
        }
    \end{vd}

    \begin{vd}
        Cho tam giác ABC. Gọi D,E,F lần lượt là trung điểm các cạnh BC, CA, AB . Chứng minh $\vec{EF} = \vec{CD}$.
        \noteImage[width=0.45\linewidth]{data//im/c3b4-v26.png}
        \loigiaiEX{
            Cách 1: Vì EF là đường trung bình của tam giác ABC nên $EF // CD$ nên
            $\vec{EF} = \frac{1}{2}\vec{CB} \Rightarrow EF = CD \Rightarrow |\vec{EF}|=|\vec{CD}|$ (1).
            Mặt khác: $\vec{EF}$ cùng hướng $\vec{CD}$ (2).
            Từ (1) và (2) ta có: $\vec{EF} = \vec{CD}$.
            Cách 2: Chứng minh EFCD là hình bình hành
            Dễ chứng minh được $\vec{EF} = \frac{1}{2}\vec{BC} = \vec{CD}$ và $EF // CD \Rightarrow EFCD$ là hình bình hành $\Rightarrow \vec{EF} = \vec{CD}$.
        }
    \end{vd}

    \begin{vd}
        Cho hình bình hành ABCD. Gọi E là điểm đối xứng C của qua D. Chứng minh rằng $\vec{AE} = \vec{BD}$.
        \noteImage[width=0.45\linewidth]{data//im/c3b4-v27.png}
        \loigiaiEX{
            Vì ABCD là hình bình hành nên ta có: $\vec{BA} = \vec{CD}$ (1).
            Ta có: E là điểm đối xứng C của qua D nên D là trung điểm của CE $\Leftrightarrow \vec{CD} = \vec{DE}$ (2).
            Từ (1) và (2) ta có: $\vec{BA} = \vec{DE} \Leftrightarrow ABDE$ là hình bình hành nên $\vec{AE} = \vec{BD}$.
        }
    \end{vd}

    \begin{vd}
        Cho $\triangle ABC$ có M, N, P lần lượt là trung điểm của các cạnh AB, BC, CA. Tìm điểm I sao cho $\vec{NP} = \vec{MI}$
        \noteImage[width=0.45\linewidth]{data//im/c3b4-v28.png}
        \loigiaiEX{
            Vì $\vec{NP} = \vec{MI}$ mà $\vec{NP} = \vec{MB}$ nên I = B .
        }
    \end{vd}

    \begin{vd}
        Cho tứ giác ABCD. Gọi M,N,P,Q lần lượt là trung điểm AB, BC, CD, DA. Chứng minh $\vec{MN} = \vec{QP}; \vec{NP} = \vec{MQ}$.
        \noteImage[width=0.45\linewidth]{data//im/c3b4-v29.png}
        \loigiaiEX{
            Ta có MN là đường trung bình tam giác $ABC \Rightarrow \vec{MN} \parallel \frac{1}{2}\vec{AC}$ và PQ là đường trung bình
            tam giác $DAC \Rightarrow \vec{PQ} \parallel \frac{1}{2}\vec{AC}$. Do đó $\vec{MN} \parallel \vec{PQ} \Rightarrow MNPQ$ là hình bình hành nên suy ra
            $\vec{MN} = \vec{QP}; \vec{NP} = \vec{MQ}$.
        }
    \end{vd}

    \begin{vd}
        Cho hình bình hành ABCD. Gọi M, N lần lượt là trung điểm của AB,DC. AN và CM lần lượt cắt BD tại E, F. Chứng minh rằng $\vec{DE} = \vec{EF} = \vec{FB}$.
        \noteImage[width=0.45\linewidth]{data//im/c3b4-v30.png}
        \loigiaiEX{
            Ta có :
            $\left\{
            \begin{array}{l}
                \vec{AM} = \vec{CN} \\
                AM // CN
            \end{array}
            \right. \Rightarrow AMCN$ là hình bình hành.

            Theo gt ta có : N là trung điểm DC và NE//CF $\Rightarrow$ NE là đường trung bình của $\triangle DFC$
            $\Rightarrow$ E là trung điểm của DF $\Rightarrow \vec{DE} = \vec{EF}$ (1).
            Tương tự ta cũng có : F là trung điểm của BE nên $\vec{EF} = \vec{FB}$ (2).
            Từ (1) và (2) ta có: $\vec{DE} = \vec{EF} = \vec{FB}$.
        }
    \end{vd}
\end{dang}

\begin{dang}{XÁC ĐỊNH ĐIỂM THOẢ ĐẲNG THỨC VECTƠ}
    \begin{pp}
        \item Sử dụng: Hai véc tơ bằng nhau khi và chỉ khi chúng cùng độ dài và cùng hướng.
    \end{pp}

    \begin{vd}
        Cho tam giác ABC. Gọi M, P, Q lần lượt là trung điểm các cạnh AB, BC, CA và N là điểm thỏa mãn $\vec{MP} = \vec{CN}$. Hãy xác định vị trí điểm N.
        \noteImage[width=0.45\linewidth]{data//im/c3b4-v31.png}
        \loigiaiEX{
            Do $\vec{MP} = \vec{CN}$ nên $|\vec{MP}| = |\vec{CN}|$ và $\vec{MP}, \vec{CN}$ cùng hướng.
            Vậy N đối xứng với Q qua C.
        }
    \end{vd}

    \begin{vd}
        Cho hình thang ABCD với đáy $BC = 2AD$. Gọi M,N,P,Q lần lượt là trung điểm của BC,MC, CD, AB và E là điểm thỏa mãn $\vec{BN} = \vec{QE}$. Xác định vị trí điểm E .
        \noteImage[width=0.45\linewidth]{data//im/c3b4-v32.png}
        \loigiaiEX{
            Ta có $\vec{BN} = \vec{QE}$ nên $|\vec{BN}| = |\vec{QE}|$ và $\vec{BN}, \vec{QE}$ cùng hướng.

            Mà $\vec{QP} = \frac{\vec{AD}+\vec{BC}}{2} = \frac{\vec{AD}+2\vec{AD}}{2} = \frac{3}{2}\vec{AD}$.
            $\vec{BN} = \vec{BM} + \vec{MN} = \frac{1}{2}\vec{BC} + \frac{1}{2}\vec{MC} = \frac{1}{2}\vec{BC} + \frac{1}{4}\vec{BC} = \frac{3}{4}\vec{BC} = \frac{3}{4}(2\vec{AD}) = \frac{3}{2}\vec{AD}$.
            Do $\vec{BN} = \vec{QE}$ và $\vec{QP} = \vec{BN}$ suy ra $\vec{QE} = \vec{QP}$ nên E=P.
        }
    \end{vd}

    \begin{vd}
        Cho tam giác ABC có trọng tâm G và N là điểm thỏa mãn $\vec{AN} = \vec{GC}$. Hãy xác định vị trí điểm N.
        \noteImage[width=0.45\linewidth]{data//im/c3b4-v33.png}
        \loigiaiEX{
            Do $\vec{AN} = \vec{GC}$ và A, C, G không thẳng hàng nên AGCN là hình bình hành.
            Vậy N đối xứng với G qua trung điểm M của AC .
        }
    \end{vd}

    \begin{vd}
        Cho hình chữ nhật ABCD, N, P lần lượt là trung điểm cạnh AD, AB và điểm M thỏa mãn $\vec{AP} = \vec{NM}$. Xác định vị trí điểm M.
        \noteImage[width=0.45\linewidth]{data//im/c3b4-v34.png}
        \loigiaiEX{
            Gọi O là tâm của hình chữ nhật ABCD $\Rightarrow \vec{AP} = \vec{NO}$.
            Mà $\vec{AP} = \vec{NM}$ suy ra $\vec{NM}=\vec{NO} \Rightarrow M=O$. Vậy M là tâm của hình chữ nhật ABCD.
        }
    \end{vd}

    \begin{vd}
        Cho hình bình hành ABCD tâm O và điểm M thỏa mãn $\vec{AO} = \vec{OM}$. Xác định vị trí điểm M.
        \noteImage[width=0.45\linewidth]{data//im/c3b4-v35.png}
        \loigiaiEX{
            Ta có $\vec{AO} = \vec{OM}$ suy ra $|\vec{AO}|=|\vec{OM}|$ và $\vec{AO}, \vec{OM}$ cùng hướng nên M =C.
        }
    \end{vd}

    \begin{vd}
        Cho $\vec{AB}$ khác $\vec{0}$ và cho điểm C. Xác định điểm D thỏa $|\vec{AB}| = |\vec{AD} - \vec{AC}|$?
        \loigiaiEX{
            Ta có $|\vec{AB}| = |\vec{AD} - \vec{AC}| \Leftrightarrow |\vec{AB}| = |\vec{CD}| \Leftrightarrow AB = CD$.
            Suy ra tập hợp các điểm D là đường tròn tâm C bán kính AB.
        }
    \end{vd}

    \begin{vd}
        Cho tam giác ABC. Xác định vị trí của điểm M sao cho $\vec{MA} – \vec{MB} + \vec{MC} = \vec{0}$
        \noteImage[width=0.45\linewidth]{data//im/c3b4-v37.png}
        \loigiaiEX{
            $\vec{MA} – \vec{MB} + \vec{MC} = \vec{0} \Leftrightarrow (\vec{MA} - \vec{MB}) + \vec{MC} = \vec{0}$
            $\Leftrightarrow \vec{BA} + \vec{MC} = \vec{0} \Leftrightarrow \vec{MC} = -\vec{BA} = \vec{AB}$.
            Vậy M thỏa mãn CBAM là hình bình hành.
        }
    \end{vd}
\end{dang}

\begin{dang}{TỔNG HỢP}
    \begin{vd}
        Cho A, B, C là ba điểm thẳng hàng, B nằm giữa A và C. Viết các cặp vectơ cùng hướng, ngược hướng trong những vectơ sau:
        $\vec{AB}, \vec{AC}, \vec{BA}, \vec{BC}, \vec{CA}, \vec{CB}$
        \noteImage[width=0.45\linewidth]{data//im/c3b4-v38.png}
        \loigiaiEX{
            Do các vectơ đều nằm trên đường thẳng AB nên các vectơ này đều cùng phương với nhau.
            Dễ thấy:
            Các vectơ $\vec{AB}, \vec{AC}, \vec{BC}$ cùng hướng (từ trái sang phải.)
            Các vectơ $\vec{BA}, \vec{CA}, \vec{CB}$ cùng hướng (từ phải sang trái.)
            Do đó, các cặp vectơ cùng hướng là:
            $\vec{AB}$ và $\vec{AC}$; $\vec{AC}$ và $\vec{BC}$; $\vec{AB}$ và $\vec{BC}$; $\vec{BA}$ và $\vec{CA}$; $\vec{BA}$ và $\vec{CB}$; $\vec{CA}$ và $\vec{CB}$.

            Các cặp vectơ ngược hướng là:
            $\vec{AB}$ và $\vec{BA}$; $\vec{AB}$ và $\vec{CA}$; $\vec{AB}$ và $\vec{CB}$.
            $\vec{AC}$ và $\vec{BA}$; $\vec{AC}$ và $\vec{CA}$; $\vec{AC}$ và $\vec{CB}$.
            $\vec{BC}$ và $\vec{BA}$; $\vec{BC}$ và $\vec{CA}$; $\vec{BC}$ và $\vec{CB}$.
        }
    \end{vd}

    \begin{vd}
        Cho đoạn thẳng MN có trung điểm là I .
        a) Viết các vectơ khác vectơ-không có điểm đầu, điểm cuối là một trong ba điểm M, N,I.
        b) vectơ nào bằng $\vec{MI}$? Bằng $\vec{NI}$ ?
        \noteImage[width=0.45\linewidth]{data//im/c3b4-v39.png}
        \loigiaiEX{
            a) Các vectơ đó là: $\vec{MI}, \vec{IM}, \vec{IN}, \vec{NI}, \vec{MN}, \vec{NM}$.
            b) Dễ thấy:
            +) vectơ $\vec{IN}$ cùng hướng với vectơ $\vec{MI}$. Hơn nữa:
            $|\vec{IN}| = IN = MI = |\vec{MI}|$
            $\Rightarrow \vec{IN} = \vec{MI}$
            +) vectơ $\vec{IM}$ cùng hướng với vectơ $\vec{NI}$. Hơn nữa:
            $|\vec{IM}| = IM = NI = |\vec{NI}|$
            $\Rightarrow \vec{IM} = \vec{NI}$
            Vậy $\vec{IN} = \vec{MI}$ và $\vec{IM} = \vec{NI}$.
        }
    \end{vd}

    \begin{vd}
        Cho hình thang ABCD có hai đáy là AB và CD. Tìm vectơ:
        a) Cùng hướng với $\vec{AB}$
        b) Ngược hướng với $\vec{AB}$
        \noteImage[width=0.45\linewidth]{data//im/c3b4-v40.png}
        \loigiaiEX{
            Giá của vectơ $\vec{AB}$ là đường thẳng AB.
            Các vectơ cùng phương với vectơ $\vec{AB}$ là: $\vec{CD}$ và $\vec{DC}$.
            a) vectơ $\vec{DC}$ cùng hướng với vectơ $\vec{AB}$.
            b) vectơ $\vec{CD}$ ngược hướng với vectơ $\vec{AB}$.
        }
    \end{vd}

    \begin{vd}
        Cho hình vuông ABCD có độ dài cạnh bằng 3cm. Tính độ dài của các vecto $\vec{AB}, \vec{AC}$.
        \noteImage[width=0.45\linewidth]{data//im/c3b4-v41.png}
        \loigiaiEX{
            Ta có: $|\vec{AB}| = AB$ và $|\vec{AC}|= AC$.
            Mà $AB = 3, AC = 3\sqrt{2}$.
            $\Rightarrow |\vec{AB}|= 3$; $|\vec{AC}|= 3\sqrt{2}$.
        }
    \end{vd}

    \begin{vd}
        Quan sát ròng rọc hoạt động khi dùng lực để kéo một đầu của ròng rọc. Chuyển động của các đoạn dây được mô tả bằng các vectơ $\vec{a},\vec{b},\vec{c}$ (hình)
        \noteImage[width=0.45\linewidth]{data//im/c3b4-v42.png}
        \loigiaiEX{
            Gọi $a, b, c$ là các đường thẳng lần lượt chứa các vectơ $\vec{a},\vec{b},\vec{c}$. Khi đó: $a,b,c$ lần lượt là giá của các vectơ $\vec{a},\vec{b},\vec{c}$.
            a) Dễ thấy: $a//b//c$
            $\Rightarrow$ Ba vectơ $\vec{a},\vec{b},\vec{c}$ cùng phương với nhau.
            Vậy các cặp vectơ cùng phương là: $\vec{a}$ và $\vec{b}$, $\vec{a}$ và $\vec{c}$, $\vec{b}$ và $\vec{c}$.
            b) Quan sát ba vectơ, ta thấy: vectơ $\vec{a}$ và $\vec{c}$ cùng hướng xuống còn vectơ $\vec{b}$ hướng lên trên.
            Vậy vectơ $\vec{a}$ và $\vec{c}$ cùng hướng, vectơ $\vec{a}$ và $\vec{b}$ ngược hướng, vecto $\vec{b}$ và $\vec{c}$ ngược hướng.
        }
    \end{vd}
\end{dang}


\chapter{vecto}

% =======================
% PHẦN I. KIẾN THỨC CƠ BẢN
% =======================
\section{Kiến thức cơ bản}

\subsection{Khái niệm Vectơ}
\begin{dn}
    Vectơ là một đoạn thẳng có hướng, nghĩa là, trong hai điểm mút của đoạn thẳng, đã chỉ rõ điểm đầu, điểm cuối.
    \noteImage[width=0.45\linewidth]{data//im/vecto-v0-01.png}
    Vectơ có điểm đầu $A$ và điểm cuối $B$ được kí hiệu là $\vec{AB}$, đọc là "vectơ $\vec{AB}$".
    Vectơ còn được kí hiệu là $\vec{a}, \vec{b}, \vec{x}, \vec{y}, \dots$ khi không cần chỉ rõ điểm đầu và điểm cuối của nó.
    Độ dài của vectơ là khoảng cách giữa điểm đầu và điểm cuối của vectơ đó.
\end{dn}

\subsection{Chú ý}
\begin{note}
    Độ dài của vectơ $\vec{AB}$ được kí hiệu là $\left|\vec{AB}\right|$, như vậy $\left|\vec{AB}\right| = AB$. Độ dài của vectơ $\vec{a}$ được kí hiệu là $\left|\vec{a}\right|$.
    Vectơ có độ dài bằng $1$ gọi là vectơ đơn vị.
\end{note}

\subsection{Hai vectơ cùng phương, cùng hướng}
\begin{dn}
    Giá của vectơ: Đường thẳng đi qua điểm đầu và điểm cuối của một vectơ được gọi là giá của vectơ đó.
    Vectơ cùng phương, vectơ cùng hướng: Hai vectơ được gọi là cùng phương nếu giá của chúng song song hoặc trùng nhau.
    Hai vectơ cùng phương thì chúng chỉ có thể cùng hướng hoặc ngược hướng.
    \noteImage[width=0.45\linewidth]{data//im/vecto-v0-02.png}
\end{dn}

\subsection{Nhận xét}
\begin{note}
    Ba điểm phân biệt $A, B, C$ thẳng hàng khi và chỉ khi hai vectơ $\vec{AB}$ và $\vec{AC}$ cùng phương.
\end{note}

\subsection{Hai vecto bằng nhau}
\begin{dn}
    Hai vectơ $\vec{a}$ và $\vec{b}$ được gọi là bằng nhau nếu chúng cùng hướng và có cùng độ dài.
    Kí hiệu $\vec{a} = \vec{b}$.
\end{dn}

\subsection{Chú ý}
\begin{note}
    Hai vectơ $\vec{a}$ và $\vec{b}$ được gọi là đối nhau nếu chúng ngược hướng và có cùng độ dài.
    Khi cho trước vectơ $\vec{a}$ và điểm $O$, thì ta luôn tìm được một điểm $A$ duy nhất sao cho $\vec{OA} = \vec{a}$.
\end{note}

\subsection{Vectơ – không}
\begin{dn}
    Vectơ – không là vectơ có điểm đầu và điểm cuối trùng nhau, ta kí hiệu là $\vec{0}$.
    Ta quy ước vectơ – không cùng phương, cùng hướng với mọi vectơ và có độ dài bằng $0$.
    Như vậy $\vec{0} = \vec{AA} = \vec{BB} = \dots$ và $\vec{MN} = \vec{0} \Leftrightarrow M = N$.
\end{dn}

% =======================
% PHẦN II. CÁC DẠNG BÀI TẬP
% =======================
\section{Các dạng bài tập}

\begin{dang}{XÁC ĐỊNH MỘT VECTƠ; PHƯƠNG, HƯỚNG CỦA VECTƠ; ĐỘ DÀI CỦA VECTO}
    \begin{pp}
        \item Xác định một vectơ và xác định sự cùng phương, cùng hướng của hai vectơ theo định nghĩa.
        \item Dựa vào các tình chất hình học của các hình đã cho biết để tính độ dài của một vectơ.
    \end{pp}

    \begin{vd}
        Với hai điểm phân biệt $A, B$ có thể xác định được bao nhiêu vectơ khác vectơ-không có điểm đầu và điểm cuối được lấy từ hai điểm trên?
        \loigiaiEX{
            Hai vectơ $\vec{AB}$ và $\vec{BA}$.
        }
    \end{vd}

    \begin{vd}
        Cho tam giác $ABC$, có thể xác định được bao nhiêu vectơ khác vectơ-không có điểm đầu và điểm cuối là các đỉnh $A, B, C$?
        \noteImage[width=0.45\linewidth]{data//im/vecto-v2.png}
        \loigiaiEX{
            Ta có $6$ vectơ: $\vec{AB}, \vec{BA}, \vec{BC}, \vec{CB}, \vec{CA}, \vec{AC}$.
        }
    \end{vd}

    \begin{vd}
        Cho hình lục giác đều $ABCDEF$ tâm $O$. Tìm số các vectơ khác vectơ - không, cùng phương với vectơ $\vec{OB}$ có điểm đầu và điểm cuối là các đỉnh của lục giác?
        \noteImage[width=0.45\linewidth]{data//im/vecto-v3.png}
        \loigiaiEX{
            Các vectơ cùng phương với vectơ $\vec{OB}$ là:
            $\vec{BE}, \vec{EB}, \vec{DC}, \vec{CD}, \vec{FA}, \vec{AF}$.
        }
    \end{vd}

    \begin{vd}
        Cho lục giác đều $ABCDEF$ tâm $O$. Tìm số các vectơ bằng $\vec{OC}$ có điểm đầu và điểm cuối là các đỉnh của lục giác?
        \loigiaiEX{
            Đó là các vectơ: $\vec{AB}, \vec{ED}$.
            \noteImage[width=0.45\linewidth]{data//im/vecto-v4.png}
        }
    \end{vd}

    \begin{vd}
        Cho hình bình hành $ABCD$ tâm $O$. Gọi $P, Q, R$ lần lượt là trung điểm của $AB, BC, AD$. Lấy $8$ điểm trên là gốc hoặc ngọn của các vectơ. Tìm số vectơ bằng với vecto $\vec{AR}$
        \noteImage[width=0.45\linewidth]{data//im/vecto-v5.png}
        \loigiaiEX{
            Có $3$ vectơ là $\vec{RD}; \vec{BQ}; \vec{QC}, \vec{PO}$.
        }
    \end{vd}

    \begin{vd}
        Cho tứ giác $ABCD$. Có bao nhiêu vectơ khác vectơ không có điểm đầu và cuối là các đỉnh của tứ giác?
        \loigiaiEX{
            Một vectơ khác vectơ không được xác định bởi $2$ điểm phân biệt. Khi có $4$ điểm $A, B, C, D$ ta có $4$ cách chọn điểm đầu và $3$ cách chọn điểm cuối. Nên ta sẽ có $3.4=12$ cách xác định số vectơ khác $\vec{0}$ thuộc $4$ điểm trên.
        }
    \end{vd}

    \begin{vd}
        Số vectơ (khác vectơ $\vec{0}$) có điểm đầu và điểm cuối lấy từ $7$ điểm phân biệt cho trước?
        \loigiaiEX{
            Một vectơ khác vectơ không được xác định bởi $2$ điểm phân biệt. Khi có $7$ điểm ta có $7$ cách chọn điểm đầu và $6$ cách chọn điểm cuối. Nên ta sẽ có $7.6 = 42$ cách xác định số vectơ khác $\vec{0}$ thuộc $7$ điểm trên.
        }
    \end{vd}

    \begin{vd}
        Trên mặt phẳng cho $6$ điểm phân biệt $A,B,C,D,E; F$. Hỏi có bao nhiêu vectơ khác vecto - không, mà có điểm đầu và điểm cuối là các điểm đã cho?
        \loigiaiEX{
            Xét tập $X = \{A, B, C, D, E ; F\}$. Với mỗi cách chọn hai phần tử của tập $X$ và sắp xếp theo một thứ tự ta được một vectơ thỏa mãn yêu cầu.
            Mỗi vectơ thỏa mãn yêu cầu tương ứng cho ta $30$ phần tử thuộc tập $X$.
            Vậy số các vectơ thỏa mãn yêu cầu bằng $30$.
        }
    \end{vd}

    \begin{vd}
        Cho $n$ điểm phân biệt. Hãy xác định số vectơ khác vectơ $\vec{0}$ có điểm đầu và điểm cuối thuộc $n$ điểm trên?
        \loigiaiEX{
            Khi có $n$ điểm, ta có $n$ cách chọn điểm đầu và $n -1$ cách chọn điểm cuối. Nên ta sẽ có $n(n-1)$ cách xác định số vectơ khác $\vec{0}$ thuộc $n$ điểm trên.
        }
    \end{vd}

    \begin{vd}
        Cho lục giác đều $ABCDEF$ tâm $O$. Số các vectơ bằng $\vec{OC}$ có điểm cuối là các đỉnh của lục giác là bao nhiêu?
        \noteImage[width=0.45\linewidth]{data//im/vecto-v10.png}
        \loigiaiEX{
            Đó là các vectơ: $\vec{AB}; \vec{ED}$.
        }
    \end{vd}

    \begin{vd}
        Cho ba điểm $M, N, P$ thẳng hàng, trong đó điểm $N$ nằm giữa hai điểm $M$ và $P$. Tìm các cặp vectơ cùng hướng?
        \noteImage[width=0.45\linewidth]{data//im/vecto-v11.png}
        \loigiaiEX{
            Các vec tơ cùng hướng là : $\vec{MN}$ và $\vec{MP}$, $\vec{MN}$ và $\vec{NP}$, $\vec{PM}$ và $\vec{PN}$, $\vec{PN}$ và $\vec{NM}$.
        }
    \end{vd}

    \begin{vd}
        Cho hình bình hành $ABCD$. Tìm vectơ khác $\vec{0}$, cùng phương với vectơ $\vec{AB}$ và có điểm đầu, điểm cuối là đỉnh của hình bình hành $ABCD$.
        \noteImage[width=0.45\linewidth]{data//im/vecto-v12.png}
        \loigiaiEX{
            Các vectơ cùng phương với $\vec{AB}$ mà thỏa mãn điều kiện đầu Câu là: $\vec{BA}, \vec{CD}, \vec{DC}$.
        }
    \end{vd}

    \begin{vd}
        Cho lục giác đều $ABCDEF$ tâm $O$. Tìm số các vectơ khác vectơ không, cùng phương với $\vec{OC}$ có điểm đầu và điểm cuối là các đỉnh của lục giác là:
        \noteImage[width=0.45\linewidth]{data//im/vecto-v13.png}
        \loigiaiEX{
            Đó là các vectơ: $\vec{AB}, \vec{BA}, \vec{DE}, \vec{ED}, \vec{FC},\vec{CF},\vec{OF},\vec{FO}$.
        }
    \end{vd}

    \begin{vd}
        Cho điểm $A$ và véctơ $\vec{a}$ khác $\vec{0}$. Tìm điểm $M$ sao cho:
        \begin{itemize}
            \item[a)] $\vec{AM}$ cùng phương với $\vec{a}$.
            \item[b)] $\vec{AM}$ cùng hướng với $\vec{a}$.
        \end{itemize}
        \noteImage[width=0.45\linewidth]{data//im/vecto-v14.png}
        \loigiaiEX{
            Gọi $\Delta$ là giá của $\vec{a}$.
            \begin{itemize}
                \item[a)] Nếu $\vec{AM}$ cùng phương với $\vec{a}$ thì đường thẳng $AM$ song song với $\Delta$. Do đó $M$ thuộc đường thẳng $m$ đi qua $A$ và song song với $\Delta$. Ngược lại, mọi điểm $M$ thuộc đường thẳng $m$ thì $\vec{AM}$ cùng phương với $\vec{a}$. Chú ý rằng nếu $A$ thuộc đường thẳng $\Delta$ thì $m$ trùng với $\Delta$.
                \item[b)] Lập luận tương tự như trên, ta thấy các điểm $M$ thuộc một nửa đường thẳng gốc $A$ của đường thẳng $m$. Cụ thể, đó là nửa đường thẳng chưa điểm $E$ sao cho $\vec{AE}$ và $\vec{a}$ cùng hướng.
            \end{itemize}
        }
    \end{vd}

    \begin{vd}
        Cho tam giác $ABC$ có trực tâm $H$. Gọi $D$ là điểm đối xứng với $B$ qua tâm $O$ của đường tròn ngoại tiếp tam giác $ABC$. Chứng minh rằng $\vec{HA} = \vec{CD}$ và $\vec{AD} = \vec{HC}$.
        \noteImage[width=0.45\linewidth]{data//im/vecto-v15.png}
        \loigiaiEX{
            Ta có $AH \perp BC$ và $DC \perp BC$ (do góc $DCB$ chắn nửa đường tròn). Suy ra $AH \parallel DC$.
            Tương tự ta cũng có $CH \parallel AD$.
            Suy ra tứ giác $ADCH$ là hình bình hành. Do đó $\vec{HA} = \vec{CD}$ và $\vec{AD} = \vec{HC}$.
        }
    \end{vd}

    \begin{vd}
        Cho tam giác $ABC$ vuông cân tại $A$, có $AB = AC = 4$. Tính $\left|\vec{BC}\right|$
        \loigiaiEX{
            Vì $\left|\vec{BC}\right| = BC = \sqrt{AB^2 + AC^2} = \sqrt{16+ 16} = 4\sqrt{2}$.
        }
    \end{vd}

    \begin{vd}
        Cho hình vuông $ABCD$ có độ dài cạnh $3$. Giá trị của $\left|\vec{AC}\right|$ là bao nhiêu?
        \noteImage[width=0.45\linewidth]{data//im/vecto-v17.png}
        \loigiaiEX{
            Vì $\left|\vec{AC}\right| = AC = 3\sqrt{2}$.
        }
    \end{vd}

    \begin{vd}
        Cho tam giác đều $ABC$ cạnh $a$. Tính $\left|\vec{CB}\right|$
        \loigiaiEX{
            Vì $\left|\vec{CB}\right| = CB = a$.
        }
    \end{vd}

    \begin{vd}
        Gọi $G$ là trọng tâm tam giác vuông $ABC$ với cạnh huyền $BC = 12$. Tính $\left|\vec{GM}\right|$ (với $M$ là trung điểm của $BC$)
        \loigiaiEX{
            Vì $\left|\vec{GM}\right| = GM = \frac{1}{3}AM = \frac{1}{3}.6=2$.
        }
    \end{vd}

    \begin{vd}
        Cho hình chữ nhật $ABCD$, có $AB = 4$ và $AC = 5$. Tìm độ dài vectơ $\vec{AC}$.
        \loigiaiEX{
            Vì $\left|\vec{AC}\right|= AC = 5$.
        }
    \end{vd}
\end{dang}

\begin{dang}{CHỨNG MINH HAI VECTƠ BẰNG NHAU}
    \begin{pp}
        \item Để chứng minh hai vectơ bằng nhau ta chứng minh chúng có cùng độ dài và cùng hướng hoặc dựa vào nhận xét nếu tứ giác $ABCD$ là hình bình hành thì $\vec{AB} = \vec{DC}$ hoặc $\vec{AD} = \vec{BC}$.
    \end{pp}

    \begin{vd}
        Cho hình vuông $ABCD$ tâm $O$. Hãy liệt kê tất cả các vectơ bằng nhau nhận đỉnh và tâm của hình vuông làm điểm đầu và điểm cuối.
        \noteImage[width=0.45\linewidth]{data//im/vecto-v21.png}
        \loigiaiEX{
            Các vectơ bằng nhau nhận đỉnh và tâm của hình vuông làm điểm đầu và điểm cuối là:
            $\vec{AB} = \vec{DC}$, $\vec{AD} = \vec{BC}$, $\vec{BA} = \vec{CD}$, $\vec{DA} = \vec{CB}$, $\vec{AO} = \vec{OC}$, $\vec{OA} = \vec{CO}$, $\vec{BO} = \vec{OD}$, $\vec{OB} = \vec{DO}$.
        }
    \end{vd}

    \begin{vd}
        Cho vectơ $\vec{AB}$ và một điểm $C$. Có bao nhiêu điểm $D$ thỏa mãn $\vec{AB} = \vec{CD}$.
        \loigiaiEX{
            Nếu $C$ nằm trên đường thẳng $AB$ thì $D$ cũng nằm trên đường thẳng $AB$.
            Nếu $C$ không nằm trên đường thẳng $AB$ thì tứ giác $ABDC$ là hình bình hành. Khi đó $D$ nằm trên đường thẳng đi qua $C$ và song song với đường thẳng $AB$.
            Do vậy, có vô số điểm $D$ thỏa mãn $\vec{AB} = \vec{CD}$.
        }
    \end{vd}

    \begin{vd}
        Cho tứ giác đều $ABCD$. Gọi $M,N,P,Q$ lần lượt là trung điểm của $AB, BC, CD, DA$. Chứng minh $\vec{MN} = \vec{QP}$.
        \noteImage[width=0.45\linewidth]{data//im/vecto-v23.png}
        \loigiaiEX{
            Ta có $MN$ là đường trung bình tam giác $ABC \Rightarrow MN \parallel AC$ và $MN = \frac{1}{2}AC$.
            $PQ$ là đường trung bình tam giác $DAC \Rightarrow PQ \parallel AC$ và $PQ = \frac{1}{2}AC$.
            Do đó $MN \parallel PQ$ và $MN = PQ$. Hơn nữa, $\vec{MN}$ và $\vec{QP}$ cùng hướng.
            Vậy $\vec{MN} = \vec{QP}$.
        }
    \end{vd}

    \begin{vd}
        Cho tứ giác $ABCD$. Điều kiện nào là điều kiện cần và đủ để $\vec{AB} = \vec{CD}$?
        \loigiaiEX{
            Ta có:
            \begin{itemize}
                \item $\vec{AB} = \vec{CD} \Rightarrow \left\{
                    \begin{array}{l}
                        AB \parallel CD \\
                        AB = CD
                    \end{array}
                \right\} \Rightarrow ABDC$ là hình bình hành.
                \item Mặt khác, $ABDC$ là hình bình hành $\Rightarrow \left\{
                    \begin{array}{l}
                        AB \parallel CD \\
                        AB = CD
                    \end{array}
                \right\} \Rightarrow \vec{AB} = \vec{CD}$.
            \end{itemize}
            Do đó, điều kiện cần và đủ để $\vec{AB} = \vec{CD}$ là $ABDC$ là hình bình hành.
        }
    \end{vd}

    \begin{vd}
        Cho hai điểm phân biệt $A, B$. Xác định điều kiện để điểm $I$ là trung điểm $AB$.
        \loigiaiEX{
            Vì $I$ là trung điểm $AB$ nên ta có $\vec{IA}+\vec{IB}=\vec{0} \Leftrightarrow \vec{IA}=-\vec{IB} \Leftrightarrow \vec{IA} = \vec{BI}$.
            Vậy điều kiện để điểm $I$ là trung điểm $AB$ là: $\vec{IA} = \vec{BI}$.
        }
    \end{vd}

    \begin{vd}
        Cho tam giác $ABC$. Gọi $D,E,F$ lần lượt là trung điểm các cạnh $BC, CA, AB$. Chứng minh $\vec{EF} = \vec{CD}$.
        \noteImage[width=0.45\linewidth]{data//im/vecto-v26.png}
        \loigiaiEX{
            Cách 1: Vì $EF$ là đường trung bình của tam giác $ABC$ nên $EF \parallel CD$ nên
            $EF = \frac{1}{2}CB \Rightarrow EF = CD \Rightarrow \left|\vec{EF}\right|=\left|\vec{CD}\right|$ (1).
            Mặt khác: $\vec{EF}$ cùng hướng $\vec{CD}$ (2).
            Từ (1) và (2) ta có: $\vec{EF} = \vec{CD}$.
            Cách 2: Chứng minh $EFCD$ là hình bình hành
            Dễ chứng minh được $EF = \frac{1}{2}BC = CD$ và $EF \parallel CD \Rightarrow EFCD$ là hình bình hành $\Rightarrow \vec{EF} =\vec{CD}$.
        }
    \end{vd}

    \begin{vd}
        Cho hình bình hành $ABCD$. Gọi $E$ là điểm đối xứng $C$ của qua $D$. Chứng minh rằng $\vec{AE} = \vec{BD}$.
        \noteImage[width=0.45\linewidth]{data//im/vecto-v27.png}
        \loigiaiEX{
            Vì $ABCD$ là hình bình hành nên ta có: $\vec{BA} = \vec{CD}$ (1).
            Ta có: $E$ là điểm đối xứng $C$ của qua $D$ nên $D$ là trung điểm cuả $CE \Leftrightarrow \vec{CD} = \vec{DE}$ (2).
            Từ (1) và (2) ta có: $\vec{BA} = \vec{DE} \Leftrightarrow ABDE$ là hình bình hành nên $\vec{AE} = \vec{BD}$.
        }
    \end{vd}

    \begin{vd}
        Cho $\triangle ABC$ có $M, N, P$ lần lượt là trung điểm của các cạnh $AB, BC, CA$. Tìm điểm $I$ sao cho $\vec{NP} = \vec{MI}$.
        \noteImage[width=0.45\linewidth]{data//im/vecto-v28.png}
        \loigiaiEX{
            Vì $\vec{NP} = \vec{MI}$ mà $\vec{NP} = \vec{MB}$ nên $I = B$.
        }
    \end{vd}

    \begin{vd}
        Cho tứ giác $ABCD$. Gọi $M,N,P,Q$ lần lượt là trung điểm $AB, BC, CD, DA$. Chứng minh $\vec{MN} = \vec{QP}$; $\vec{NP} = \vec{MQ}$.
        \noteImage[width=0.45\linewidth]{data//im/vecto-v29.png}
        \loigiaiEX{
            Ta có $MN$ là đường trung bình tam giác $ABC \Rightarrow MN \parallel AC$ và $MN = \frac{1}{2}AC$.
            $PQ$ là đường trung bình tam giác $DAC \Rightarrow PQ \parallel AC$ và $PQ = \frac{1}{2}AC$.
            Do đó $MN \parallel PQ \Rightarrow MNPQ$ là hình bình hành nên suy ra $\vec{MN} = \vec{QP}$; $\vec{NP} = \vec{MQ}$.
        }
    \end{vd}

    \begin{vd}
        Cho hình bình hành $ABCD$. Gọi $M, N$ lần lượt là trung điểm của $AB,DC$. $AN$ và $CM$ lần lượt cắt $BD$ tại $E, F$. Chứng minh rằng $DE = EF = FB$.
        \noteImage[width=0.45\linewidth]{data//im/vecto-v30.png}
        \loigiaiEX{
            Ta có :
            \[
            \left.\begin{array}{l}
            AM = CN \\
            AM \parallel CN
            \end{array}\right\} \Rightarrow AMCN \text{ là hình bình hành.}
            \]
            Theo gt ta có : $N$ là trung điểm $DC$ và $NE \parallel CF \Rightarrow NE$ là đường trung bình của $\triangle DFC$
            $\Rightarrow E$ là trung điểm của $DF \Rightarrow DE = EF$ (1).
            Tương tự ta cũng có : $F$ là trung điểm của $BE$ nên $EF = FB$ (2).
            Từ (1) và (2) ta có: $DE = EF = FB$.
        }
    \end{vd}
\end{dang}

\begin{dang}{XÁC ĐỊNH ĐIỂM THOẢ ĐẲNG THỨC VECTƠ}
    \begin{pp}
        \item Sử dụng: Hai véc tơ bằng nhau khi và chỉ khi chúng cùng độ dài và cùng hướng.
    \end{pp}

    \begin{vd}
        Cho tam giác $ABC$. Gọi $M, P, Q$ lần lượt là trung điểm các cạnh $AB, BC, CA$ và $N$ là điểm thỏa mãn $\vec{MP} = \vec{CN}$. Hãy xác định vị trí điểm $N$.
        \noteImage[width=0.45\linewidth]{data//im/vecto-v31.png}
        \loigiaiEX{
            Do $\vec{MP} = \vec{CN}$ nên $MP = CN$ và $\vec{MP}, \vec{CN}$ cùng hướng.
            Vậy $N$ đối xứng với $Q$ qua $C$.
        }
    \end{vd}

    \begin{vd}
        Cho hình thang $ABCD$ với đáy $BC = 2AD$. Gọi $M,N,P,Q$ lần lượt là trung điểm của $BC,MC, CD, AB$ và $E$ là điểm thỏa mãn $\vec{BN} = \vec{QE}$. Xác định vị trí điểm $E$.
        \noteImage[width=0.45\linewidth]{data//im/vecto-v32.png}
        \loigiaiEX{
            Ta có $\vec{BN} = \vec{QE}$ nên $BN = QE$ và $\vec{BN}, \vec{QE}$ cùng hướng.
            Mà $QP = \frac{AD+BC}{2} = \frac{3}{2}AD = BN$, suy ra $\vec{QP}=\vec{BN}$ nên $E=P$.
        }
    \end{vd}

    \begin{vd}
        Cho tam giác $ABC$ có trọng tâm $G$ và $N$ là điểm thỏa mãn $\vec{AN} = \vec{GC}$. Hãy xác định vị trí điểm $N$.
        \noteImage[width=0.45\linewidth]{data//im/vecto-v33.png}
        \loigiaiEX{
            Do $\vec{AN} = \vec{GC}$ và $A, C, G$ không thẳng hàng nên $AGCN$ là hình bình hành.
            Vậy $N$ đối xứng với $G$ qua trung điểm $M$ của $AC$.
        }
    \end{vd}

    \begin{vd}
        Cho hình chữ nhật $ABCD$, $N, P$ lần lượt là trung điểm cạnh $AD, AB$ và điểm $M$ thỏa mãn $\vec{AP} = \vec{NM}$. Xác định vị trí điểm $M$.
        \noteImage[width=0.45\linewidth]{data//im/vecto-v34.png}
        \loigiaiEX{
            Gọi $O$ là tâm của hình chữ nhật $ABCD \Rightarrow \vec{AP} = \vec{NO}$.
            Mà $\vec{AP} = \vec{NM}$ suy ra $\vec{NM}=\vec{NO} \Rightarrow M=O$. Vậy $M$ là tâm của hình chữ nhật $ABCD$.
        }
    \end{vd}

    \begin{vd}
        Cho hình bình hành $ABCD$ tâm $O$ và điểm $M$ thỏa mãn $\vec{AO} = \vec{OM}$. Xác định vị trí điểm $M$.
        \noteImage[width=0.45\linewidth]{data//im/vecto-v35.png}
        \loigiaiEX{
            Ta có $\vec{AO} = \vec{OM}$ suy ra $AO=OM$ và $\vec{AO}, \vec{OM}$ cùng hướng nên $M = C$.
        }
    \end{vd}

    \begin{vd}
        Cho $\vec{AB}$ khác $\vec{0}$ và cho điểm $C$. Xác định điểm $D$ thỏa $\left|\vec{AB}\right| = \left|\vec{AD} - \vec{AC}\right|$?
        \loigiaiEX{
            Ta có $\left|\vec{AB}\right| = \left|\vec{AD} - \vec{AC}\right| \Leftrightarrow \left|\vec{AB}\right| = \left|\vec{CD}\right| \Leftrightarrow AB = CD$.
            Suy ra tập hợp các điểm $D$ là đường tròn tâm $C$ bán kính $AB$.
        }
    \end{vd}

    \begin{vd}
        Cho tam giác $ABC$. Xác định vị trí của điểm $M$ sao cho $\vec{MA} - \vec{MB} + \vec{MC} = \vec{0}$.
        \noteImage[width=0.45\linewidth]{data//im/vecto-v37.png}
        \loigiaiEX{
            $\vec{MA} - \vec{MB} + \vec{MC} = \vec{0} \Leftrightarrow \vec{BA} + \vec{MC} = \vec{0} \Leftrightarrow \vec{CM} = \vec{BA}$.
            Vậy $M$ thỏa mãn $CBAM$ là hình bình hành.
        }
    \end{vd}
\end{dang}

\begin{dang}{TỔNG HỢP}
    \begin{vd}
        Cho $A, B, C$ là ba điểm thẳng hàng, $B$ nằm giữa $A$ và $C$. Viết các cặp vectơ cùng hướng, ngược hướng trong những vectơ sau:
        $\vec{AB}, \vec{AC}, \vec{BA}, \vec{BC},\vec{CA},\vec{CB}$
        \noteImage[width=0.45\linewidth]{data//im/vecto-v38.png}
        \loigiaiEX{
            Do các vectơ đều nằm trên đường thẳng $AB$ nên các vectơ này đều cùng phương với nhau.
            Dễ thấy:
            Các vectơ $\vec{AB}, \vec{AC}, \vec{BC}$ cùng hướng (từ trái sang phải.)
            Các vectơ $\vec{BA}, \vec{CA}, \vec{CB}$ cùng hướng (từ phải sang trái.)
            Do đó, các cặp vectơ cùng hướng là:
            $\vec{AB}$ và $\vec{AC}$; $\vec{AC}$ và $\vec{BC}$; $\vec{AB}$ và $\vec{BC}$; $\vec{BA}$ và $\vec{CA}$; $\vec{BA}$ và $\vec{CB}$; $\vec{CA}$ và $\vec{CB}$.
            Các cặp vectơ ngược hướng là:
            $\vec{AB}$ và $\vec{BA}$; $\vec{AB}$ và $\vec{CA}$; $\vec{AB}$ và $\vec{CB}$
            $\vec{AC}$ và $\vec{BA}$; $\vec{AC}$ và $\vec{CA}$; $\vec{AC}$ và $\vec{CB}$;
            $\vec{BC}$ và $\vec{BA}$; $\vec{BC}$ và $\vec{CA}$; $\vec{BC}$ và $\vec{CB}$.
        }
    \end{vd}

    \begin{vd}
        Cho đoạn thẳng $MN$ có trung điểm là $I$.
        \begin{itemize}
            \item[a)] Viết các vectơ khác vectơ-không có điểm đầu, điểm cuối là một trong ba điểm $M, N,I$.
            \item[b)] vectơ nào bằng $\vec{MI}$? Bằng $\vec{NI}$?
        \end{itemize}
        \noteImage[width=0.45\linewidth]{data//im/vecto-v39.png}
        \loigiaiEX{
            \begin{itemize}
                \item[a)] Các vectơ đó là: $\vec{MI}, \vec{IM}, \vec{IN}, \vec{NI}, \vec{MN}, \vec{NM}$.
                \item[b)] Dể thấy:
                \begin{itemize}
                    \item[+] vectơ $\vec{IN}$ cùng hướng với vectơ $\vec{MI}$. Hơn nữa:
                    $\left|\vec{IN}\right| = IN = MI = \left|\vec{MI}\right|$
                    $\Rightarrow \vec{IN} = \vec{MI}$
                    \item[+] vectơ $\vec{IM}$ cùng hướng với vectơ $\vec{NI}$. Hơn nữa:
                    $\left|\vec{IM}\right| = IM = NI = \left|\vec{NI}\right|$
                    $\Rightarrow \vec{IM} = \vec{NI}$
                \end{itemize}
                Vậy $\vec{IN} = \vec{MI}$ và $\vec{IM} = \vec{NI}$.
            \end{itemize}
        }
    \end{vd}

    \begin{vd}
        Cho hình thang $ABCD$ có hai đáy là $AB$ và $CD$. Tìm vectơ:
        \begin{itemize}
            \item[a)] Cùng hướng với $\vec{AB}$
            \item[b)] Ngược hướng với $\vec{AB}$
        \end{itemize}
        \noteImage[width=0.45\linewidth]{data//im/vecto-v40.png}
        \loigiaiEX{
            Giá của vectơ $\vec{AB}$ là đường thẳng $AB$.
            Các vectơ cùng phương với vectơ $\vec{AB}$ là: $\vec{CD}$ và $\vec{DC}$.
            \begin{itemize}
                \item[a)] vectơ $\vec{DC}$ cùng hướng với vectơ $\vec{AB}$.
                \item[b)] vectơ $\vec{CD}$ ngược hướng với vectơ $\vec{AB}$.
            \end{itemize}
        }
    \end{vd}

    \begin{vd}
        Cho hình vuông $ABCD$ có độ dài cạnh bằng $3cm$. Tính độ dài của các vecto $\vec{AB}, \vec{AC}$.
        \noteImage[width=0.45\linewidth]{data//im/vecto-v41.png}
        \loigiaiEX{
            Ta có: $\left|\vec{AB}\right| = AB$ và $\left|\vec{AC}\right|= AC$.
            Mà $AB = 3, AC = 3\sqrt{2}$.
            $\Rightarrow \left|\vec{AB}\right|= 3$; $\left|\vec{AC}\right|= 3\sqrt{2}$.
        }
    \end{vd}

    \begin{vd}
        Quan sát ròng rọc hoạt động khi dùng lực để kéo một đầu của ròng rọc. Chuyển động của các đoạn dây được mô tả bằng các vectoo $\vec{a},\vec{b},\vec{c}$ (hình)
        \noteImage[width=0.45\linewidth]{data//im/vecto-v42.png}
        \begin{itemize}
            \item[a)] Hãy chỉ ra các cặp vectơ cùng phương.
            \item[b)] Trong các cặp vectơ đó, cho biết chúng cùng hướng hay ngược hướng.
        \end{itemize}
        \loigiaiEX{
            Gọi $a, b, c$ là các đường thẳng lần lượt chứa các vectơ $\vec{a},\vec{b},\vec{c}$. Khi đó: $a,b,c$ lần lượt là giá của các vecto $\vec{a},\vec{b},\vec{c}$.
            \begin{itemize}
                \item[a)] Dễ thấy: $a \parallel b \parallel c$.
                $\Rightarrow$ Ba vectơ $\vec{a},\vec{b},\vec{c}$ cùng phương với nhau.
                Vậy các cặp vectơ cùng phương là: $\vec{a}$ và $\vec{b}$, $\vec{a}$ và $\vec{c}$, $\vec{b}$ và $\vec{c}$.
                \item[b)] Quan sát ba vectơ, ta thấy: vectơ $\vec{a}$ và $\vec{c}$ cùng hướng xuống còn vectơ $\vec{b}$ hướng lên trên.
                Vậy vectơ $\vec{a}$ và $\vec{c}$ cùng hướng, vectơ $\vec{a}$ và $\vec{b}$ ngược hướng, vecto $\vec{b}$ và $\vec{c}$ ngược hướng.
            \end{itemize}
        }
    \end{vd}
\end{dang}


<section>
    <h2>ĐƯỜNG TIỆM CẬN CỦA ĐỒ THỊ HÀM SỐ</h2>
    <section>
        <h3>ĐƯỜNG TIỆM CẬN NGANG</h3>
        <div class="box definition">
            <h3>Định nghĩa</h3>
            <p>Đường thẳng \(y=y_0\) là đường tiệm cận ngang của đồ thị hàm số \(y=f(x)\) nếu thỏa mãn một trong các điều kiện sau:</p>
            \[\lim_{x \to +\infty} f(x) = y_0 \quad \text{hoặc} \quad \lim_{x \to -\infty} f(x) = y_0\]
        </div>
        <div class="box note">
            <h3>Lưu ý</h3>
            <p>Để tìm tiệm cận ngang của đồ thị hàm số \(y=f(x)\), ta tính giới hạn của hàm số tại \(\pm\infty\). Nếu kết quả là một hằng số thì đó là giá trị của tiệm cận ngang.</p>
        </div>
    </section>
    <section>
        <h3>ĐƯỜNG TIỆM CẬN ĐỨNG</h3>
        <div class="box definition">
            <h3>Định nghĩa</h3>
            <p>Đường thẳng \(x=x_0\) là đường tiệm cận đứng của đồ thị hàm số \(y=f(x)\) nếu thỏa mãn một trong các điều kiện sau:</p>
            \[\lim_{x \to x_0^+} f(x) = \pm\infty \quad \text{hoặc} \quad \lim_{x \to x_0^-} f(x) = \pm\infty\]
        </div>
    </section>
    <section>
        <section>
            <div class="box example">
                <h3>Ví dụ:</h3>
                <p>Tìm các đường tiệm cận của đồ thị hàm số \(y = \frac{2x+1}{x-1}\).</p>
            </div>
        </section>
        <section>
            <div class="box solution">
                <h4>📝 Lời giải</h4>
                <p>- Ta có \(\lim_{x \to \pm\infty} \frac{2x+1}{x-1} = 2\). Vậy \(y=2\) là tiệm cận ngang.</p>
                <p>- Ta có \(\lim_{x \to 1^+} \frac{2x+1}{x-1} = +\infty\) và \(\lim_{x \to 1^-} \frac{2x+1}{x-1} = -\infty\). Vậy \(x=1\) là tiệm cận đứng.</p>
            </div>
            <img src="data/im/71.png" alt="minh họa">
        </section>
    </section>
</section>

<section>
    <h2>Phương pháp chung để khảo sát và vẽ đồ thị hàm số</h2>
    <section>
        <div class="box procedure">
            <h3>Bước 1: Tìm tập xác định \(D\) của hàm số.</h3>
        </div>
        <div class="box procedure">
            <h3>Bước 2: Sự biến thiên của hàm số.</h3>
            <ul>
                <li>Tìm đạo hàm \(y'\). Tìm các điểm mà đạo hàm bằng 0 hoặc không tồn tại.</li>
                <li>Tính các giới hạn và tìm tiệm cận (nếu có) của đồ thị hàm số.</li>
                <li>Lập bảng biến thiên của hàm số.</li>
                <li>Kết luận về các khoảng đơn điệu, các điểm cực trị của hàm số (nếu có).</li>
            </ul>
        </div>
    </section>
    <section>
        <div class="box procedure">
            <h3>Bước 3: Lập bảng giá trị và vẽ đồ thị hàm số.</h3>
            <ul>
                <li>Vẽ các đường tiệm cận (nếu có) của đồ thị hàm số.</li>
                <li>Ta có thể tìm một số điểm đặc biệt thuộc đồ thị hàm số như giao điểm của đồ thị với trục hoành, trục tung (nếu việc tìm này đơn giản).</li>
            </ul>
        </div>
    </section>
</section>

<section>
    <h2>Khảo sát và vẽ đồ thị một số hàm số thường gặp</h2>
    <section>
        <h3>Hàm số bậc ba \(y = ax^3 + bx^2 + cx + d\) (\(a \ne 0\))</h3>
        <div class="box procedure">
            <ul>
                <li><strong>Bước 1:</strong> Tập xác định: \(D=\mathbb{R}\).</li>
                <li><strong>Bước 2:</strong> Sự biến thiên.
                    <ul>
                        <li>\(y' = 3ax^2+2bx+c\); cho \(y'=0\) và tìm nghiệm \(x_1, x_2\) (nếu có).</li>
                        <li>Tính các giới hạn tại vô cực của hàm số: \(\lim_{x \to +\infty} y\) và \(\lim_{x \to -\infty} y\).</li>
                        <li>Lập bảng biến thiên của hàm số.</li>
                        <li>Kết luận khoảng đồng biến, nghịch biến; các điểm cực trị của hàm số.</li>
                    </ul>
                </li>
                <li><strong>Bước 3:</strong> Vẽ đồ thị hàm số.
                    <ul>
                        <li>Lập bảng giá trị, tìm một số điểm đặc biệt thuộc đồ thị trước khi vẽ hình.</li>
                    </ul>
                </li>
            </ul>
        </div>
        <div class="box note">
            <h3>Lưu ý</h3>
            <p><strong>Lưu ý:</strong> Đồ thị hàm số có tâm đối xứng là (\(x_0; y_0\)), trong đó \(x_0\) là nghiệm của phương trình \(y''=0\).</p>
        </div>
        <div class="box formula">
            <h3>Một số ghi nhớ</h3>
            <p><strong>Nhận xét:</strong> Đồ thị hàm số bậc ba được minh họa như hình sau:</p>
            <img src="data//im/104.png" alt="minh họa">
        </div>
    </section>

    <section>
        <h3>Hàm số phân thức hữu tỉ \(y=\frac{ax+b}{cx+d}\) (\(c \ne 0, ad-bc \ne 0\))</h3>
        <div class="box procedure">
            <ul>
                <li><strong>Bước 1:</strong> Tập xác định: \(D = \mathbb{R} \setminus \{-\frac{d}{c}\}\).</li>
                <li><strong>Bước 2:</strong> Sự biến thiên.
                    <ul>
                        <li>\(y' = \frac{ad-bc}{(cx+d)^2}\); \(y'\) chỉ mang một dấu nên hàm số luôn đồng biến (hoặc luôn nghịch biến) trên mỗi khoảng xác định của nó.</li>
                        <li>Hàm số không có điểm cực trị.</li>
                        <li>Tính các giới hạn: \(\lim_{x \to (-\frac{d}{c})^+} y\), \(\lim_{x \to (-\frac{d}{c})^-} y\) và \(\lim_{x \to \pm\infty} y = \frac{a}{c}\). Kết luận về tiệm cận đứng \(x=-\frac{d}{c}\) và tiệm cận ngang \(y=\frac{a}{c}\) của đồ thị hàm số.</li>
                        <li>Lập bảng biến thiên của hàm số.</li>
                    </ul>
                </li>
                <li><strong>Bước 3:</strong> Vẽ đồ thị hàm số.
                    <ul>
                        <li>Vẽ đường tiệm cận đứng, tiệm cận ngang.</li>
                        <li>Tìm các điểm đặc biệt thuộc đồ thị và vẽ đồ thị hàm số.</li>
                    </ul>
                </li>
            </ul>
        </div>
        <div class="box formula">
            <h3>Một số ghi nhớ</h3>
            <ul>
                <li>Tâm đối xứng của đồ thị hàm số là giao điểm của hai đường tiệm cận: \(I(-\frac{d}{c}; \frac{a}{c})\).</li>
                <li>Trục đối xứng của đồ thị hàm số là hai đường phân giác góc tạo bởi tiệm cận đứng và tiệm cận ngang của đồ thị hàm số đó.</li>
            </ul>
        </div>
        <div class="box note">
            <h3>Lưu ý</h3>
            <p>Đồ thị hàm số \(y=\frac{ax+b}{cx+d}\) (\(c \ne 0, ad-bc \ne 0\)) được minh họa như hình sau:</p>
            <img src="data//im/105.png" alt="minh họa">
        </div>
    </section>

    <section>
        <h3>Hàm số phân thức hữu tỉ \(y=\frac{ax^2+bx+c}{dx+e}\) (\(a, d \ne 0\))</h3>
        <div class="box procedure">
            <ul>
                <li><strong>Bước 1:</strong> Tập xác định: \(D=\mathbb{R} \setminus \{-\frac{e}{d}\}\).</li>
                <li><strong>Bước 2:</strong> Sự biến thiên.
                    <ul>
                        <li>\(y' = \frac{adx^2+2aex+be-cd}{(dx+e)^2}\); cho \(y'=0\) và tìm nghiệm \(x_1, x_2\) (nếu có).</li>
                        <li>Tính các giới hạn: \(\lim_{x \to (-\frac{e}{d})^+} y\), \(\lim_{x \to (-\frac{e}{d})^-} y\) và \(\lim_{x \to \pm\infty} [f(x)-(\alpha x + \beta)]=0\). Kết luận về tiệm cận đứng \(x=-\frac{e}{d}\) và tiệm xiên \(y=\alpha x + \beta\) của đồ thị hàm số.</li>
                        <li>Lập bảng biến thiên của hàm số và kết luận về các khoảng đơn điệu, cực trị của hàm số (nếu có).</li>
                    </ul>
                </li>
                <li><strong>Bước 3:</strong> Vẽ đồ thị hàm số.
                    <ul>
                        <li>Vẽ đường tiệm cận đứng, tiệm cận xiên của đồ thị hàm số.</li>
                        <li>Tìm các điểm đặc biệt thuộc đồ thị và vẽ đồ thị hàm số.</li>
                    </ul>
                </li>
            </ul>
        </div>
        <div class="box note">
            <h3>Lưu ý</h3>
            <ul>
                <li>Tâm đối xứng của đồ thị hàm số là giao điểm của hai đường tiệm cận đứng và tiệm cận xiên.</li>
                <li>Trục đối xứng của đồ thị hàm số là hai đường phân giác góc tạo bởi tiệm cận đứng và tiệm cận xiên của đồ thị hàm số đó.</li>
            </ul>
        </div>
        <div class="box note">
            <h3>Lưu ý</h3>
            <p>Đồ thị hàm số \(y=\frac{ax^2+bx+c}{dx+e}\) (\(a, d \ne 0\)) được minh họa như hình sau:</p>
            <img src="data//im/106.png" alt="minh họa">
        </div>
    </section>
</section>

<section>
    <h2>Nhận diện đồ thị hàm bậc 3</h2>
    <section>
        <div class="box formula">
            <h3>Một số ghi nhớ</h3>
            <p>Hàm số bậc ba có dạng \(y = ax^3 + bx^2 + cx + d\) với \(a \ne 0\)</p>
            <p>☞ Đạo hàm: \(y' = 3ax^2 + 2bx + c\) ; \(y'' = 6ax + 2b\)</p>
        </div>
        <h3>Xác định dấu của a</h3>
        <div class="box procedure">
            <ul>
                <li>Nhìn vào góc phải đồ thị, ta thấy đồ thị đi lên <strong>trên</strong>, tức là \(\lim_{x \to +\infty} y = +\infty\), ta có \(\boxed{a > 0}\).</li>
                <li>Ngược lại nhánh phải đồ thị đi <strong>xuống dưới</strong>, tức là \(\lim_{x \to +\infty} y = -\infty\), ta có \(\boxed{a < 0}\).</li>
            </ul>
        </div>
        <img src="data//im/107.png" alt="minh họa">
    </section>

    <section>
        <h3>Xác định dấu của d</h3>
        <div class="box procedure">
            <ul>
                <li>Xét giao điểm của đồ thị hàm số với trục tung: \(\begin{cases} x = 0 \\ y = d \end{cases}\).</li>
                <li>Giao điểm của đồ thị với trục tung nằm <strong>trên gốc tọa độ</strong> \(O \Rightarrow \boxed{d > 0}\).</li>
                <li>Giao điểm của đồ thị với trục tung nằm <strong>dưới gốc tọa độ</strong> \(O \Rightarrow \boxed{d < 0}\).</li>
                <li>Giao điểm của đồ thị với trục tung <strong>trùng với gốc tọa độ</strong> \(O \Rightarrow \boxed{d = 0}\).</li>
            </ul>
        </div>
        <img src="data//im/108.png" alt="minh họa">
    </section>

    <section>
        <h3>Xác định dấu của b</h3>
        <div class="box procedure">
            <ul>
                <li>Xét tọa độ điểm uốn (tâm đối xứng) của đồ thị hàm số là \(I(x_I; y_I)\) với \(x_I = -\dfrac{b}{3a}\).</li>
                <li>Điểm uốn \(I\) nằm <strong>bên phải trục tung</strong> \(Oy \Rightarrow -\dfrac{b}{3a} > 0 \Rightarrow \dfrac{b}{a} < 0 \Rightarrow \boxed{ab < 0}\).</li>
                <li>Điểm uốn \(I\) nằm <strong>bên trái trục tung</strong> \(Oy \Rightarrow -\dfrac{b}{3a} < 0 \Rightarrow \dfrac{b}{a} > 0 \Rightarrow \boxed{ab > 0}\).</li>
                <li>Điểm uốn \(I\) <strong>thuộc trục tung</strong> \(Oy\) (tức là hai điểm cực trị cách đều trục tung) \(\Rightarrow \boxed{b = 0}\).</li>
            </ul>
        </div>
        <img src="data//im/109.png" alt="minh họa">
        <div class="box note">
            <h3>Lưu ý</h3>
            <p>Trong trường hợp đồ thị hàm số có hai điểm cực trị, ta có thể sử dụng <strong>định lí vi-ét</strong> để xét dấu của \(b\) (sau khi biết dấu của a), ta có: \(x_1 + x_2 = -\dfrac{B}{A} = -\dfrac{2b}{3a}\). Tùy vào tổng này âm, dương hoặc bằng 0 mà ta kết luận được dấu của b.</p>
        </div>
    </section>

    <section>
        <h3>Xác định dấu của c</h3>
        <div class="box procedure">
            <ul>
                <li>Hai điểm cực trị nằm <strong>cùng phía</strong> với trục tung \(Oy\): \(x_1x_2 = \dfrac{c}{3a} > 0 \Rightarrow \boxed{ac > 0}\).</li>
                <li>Hai điểm cực trị nằm <strong>khác phía</strong> với trục tung \(Oy\): \(x_1x_2 = \dfrac{c}{3a} < 0 \Rightarrow \boxed{ac < 0}\).</li>
            </ul>
        </div>
        <div class="box note">
            <h3>Lưu ý</h3>
            <p>Ngoài các quy tắc xét dấu hệ số hàm bậc ba như trên, ta còn có thể đánh giá đồ thị hàm số theo hai trường hợp sau:</p>
            <ul>
                <li>Đồ thị hàm số bậc ba có hai điểm cực trị \(\Leftrightarrow \begin{cases} a \ne 0 \\ \Delta > 0 \end{cases}\).</li>
                <li>Đồ thị hàm số bậc ba không có điểm cực trị \(\Leftrightarrow \begin{cases} a \ne 0 \\ \Delta \le 0 \end{cases}\).</li>
            </ul>
        </div>
        <img src="data//im/110.png" alt="minh họa">
    </section>

    <section>
        <section>
            <div class="box example">
                <h3>Ví dụ:</h3>
                <p>Đồ thị của hàm số nào dưới đây có dạng như đường cong trong hình bên?</p>
                <ol>
                    <li>\(A.\) \(y = x^3 - 3x\).</li>
                    <li>\(B.\) \(y = -x^3 + 3x\).</li>
                    <li>\(C.\) \(y = x^4 - 2x^2\).</li>
                    <li>\(D.\) \(y = -x^4 + 2x^2\).</li>
                </ol>
                <img src="data//im/111.png" alt="minh họa">
            </div>
        </section>
        <section>
            <div class="box solution">
                <h4>📝 Lời giải</h4>
                <p><strong>Chọn A</strong></p>
                <p>Dạng đồ thị này là của hàm số bậc ba nên loại C, D.</p>
                <p>Nhánh phải đồ thị đi lên, tức là \(\lim_{x \to +\infty} y = +\infty\). Suy ra \(a>0\).</p>
            </div>
        </section>
    </section>

    <section>
        <section>
            <div class="box example">
                <h3>Ví dụ:</h3>
                <p>Hình vẽ sau đây là đồ thị của một trong bốn hàm số cho ở các đáp án A, B, C, D. Hỏi đó là hàm số nào?</p>
                <ol>
                    <li>\(A.\) \(y = x^3 + 2x + 1\).</li>
                    <li>\(B.\) \(y = x^3 - 2x^2 + 1\).</li>
                    <li>\(C.\) \(y = x^3 - 2x + 1\).</li>
                    <li>\(D.\) \(y = -x^3 + 2x + 1\).</li>
                </ol>
                <img src="data//im/112.png" alt="minh họa">
            </div>
        </section>
        <section>
            <div class="box solution">
                <h4>📝 Lời giải</h4>
                <p><strong>Chọn C</strong></p>
                <p>Nhánh phải đồ thị đi lên nên \(\lim_{x \to +\infty} y = +\infty\). Suy ra \(a>0\). <strong>Loại D</strong></p>
                <p>Hàm số có hai điểm cực trị nên \(y' = 0\) có hai nghiệm phân biệt, ta <strong>loại A</strong>.</p>
                <p>Xét B: \(y' = 3x^2 - 4x\); \(y' = 0 \Rightarrow \begin{cases} x = 0 \\ x = \dfrac{4}{3} \end{cases}\) (loại). <strong>Loại B</strong></p>
                <p>Xét C: \(y' = 3x^2 - 2\); \(y' = 0 \Rightarrow x = \pm \sqrt{\dfrac{2}{3}}\) (nhận).</p>
            </div>
        </section>
    </section>

    <section>
        <section>
            <div class="box example">
                <h3>Ví dụ:</h3>
                <p>Cho hàm số \(y = ax^3 + bx^2 + cx + d\) có đồ thị như hình vẽ bên. Mệnh đề nào dưới đây đúng?</p>
                <ol>
                    <li>\(A.\) \(a < 0, b > 0, c > 0, d < 0\).</li>
                    <li>\(B.\) \(a < 0, b < 0, c > 0, d < 0\).</li>
                    <li>\(C.\) \(a > 0, b < 0, c < 0, d > 0\).</li>
                    <li>\(D.\) \(a < 0, b > 0, c < 0, d < 0\).</li>
                </ol>
                <img src="data//im/113.png" alt="minh họa">
            </div>
        </section>
        <section>
            <div class="box solution">
                <h4>📝 Lời giải</h4>
                <p><strong>Chọn A</strong></p>
                <p>Nhánh phải đồ thị đi xuống nên \(\lim_{x \to +\infty} y = -\infty \Rightarrow \boxed{a < 0}\).</p>
                <p>Giao điểm của đồ thị với trục tung: \(\begin{cases} x = 0 \\ y = d < 0 \end{cases}\) (do giao điểm này nằm dưới gốc tọa độ).</p>
                <p>Ta có: \(y' = 3ax^2 + 2bx + c \Rightarrow y'' = 6ax + 2b = 0 \Rightarrow x = -\dfrac{b}{3a} = x_I\) (hoành độ tâm đối xứng).</p>
                <p>Vì tâm đối xứng đồ thị nằm bên phải trục Oy nên \(x_I = -\dfrac{b}{3a} > 0 \Rightarrow \dfrac{b}{a} < 0 \Rightarrow \boxed{b > 0}\).</p>
                <p>Hai điểm cực trị nằm hai phía Oy nên \(x_1x_2 < 0 \Rightarrow \dfrac{c}{3a} < 0 \Rightarrow \dfrac{c}{a} < 0 \Rightarrow \boxed{c > 0}\).</p>
            </div>
        </section>
    </section>

    <section>
        <section>
            <div class="box example">
                <h3>Ví dụ:</h3>
                <p>Cho đường cong \((C): y = ax^3 + bx^2 + cx + d\) có đồ thị như hình bên. Khẳng định nào sau đây là đúng?</p>
                <ol>
                    <li>\(A.\) \(a > 0, b < 0, c < 0, d < 0\).</li>
                    <li>\(B.\) \(a > 0, b > 0, c < 0, d > 0\).</li>
                    <li>\(C.\) \(a < 0, b > 0, c > 0, d < 0\).</li>
                    <li>\(D.\) \(a > 0, b > 0, c < 0, d < 0\).</li>
                </ol>
                <img src="data//im/114.png" alt="minh họa">
            </div>
        </section>
        <section>
            <div class="box solution">
                <h4>📝 Lời giải</h4>
                <p><strong>Chọn D</strong></p>
                <p>Vì nhánh phải đồ thị đi lên nên \(\lim_{x \to +\infty} y = +\infty \Rightarrow \boxed{a > 0}\).</p>
                <p>Giao điểm của đồ thị với trục tung: \(\begin{cases} x = 0 \\ y = d < 0 \end{cases}\) (do giao điểm này nằm dưới gốc tọa độ).</p>
                <p>Ta có: \(y' = 3ax^2 + 2bx + c \Rightarrow y'' = 6ax + 2b = 0 \Rightarrow x = -\dfrac{b}{3a} = x_I\) (hoành độ tâm đối xứng).</p>
                <p>Vì tâm đối xứng đồ thị nằm bên trái trục Oy nên \(x_I = -\dfrac{b}{3a} < 0 \Rightarrow \dfrac{b}{a} > 0 \Rightarrow \boxed{b > 0}\).</p>
                <p>Hai điểm cực trị nằm hai phía Oy nên \(x_1x_2 < 0 \Rightarrow \dfrac{c}{3a} < 0 \Rightarrow \dfrac{c}{a} < 0 \Rightarrow \boxed{c < 0}\).</p>
            </div>
        </section>
    </section>
</section>

<section>
    <h2>Nhận diện đồ thị hàm nhất biến</h2>
    <section>
        <div class="box formula">
            <h3>Một số ghi nhớ</h3>
            <p>Hàm số nhất biến có dạng \(y = \dfrac{ax+b}{cx+d}\) với \(c \ne 0, ad - bc \ne 0\)</p>
        </div>
        <h3>Tiệm cận đứng: \(x = -\dfrac{d}{c}\)</h3>
        <div class="box procedure">
            <ul>
                <li>Nếu tiệm cận đứng nằm <strong>bên phải</strong> Oy thì \(-\dfrac{d}{c} > 0 \Rightarrow \dfrac{d}{c} < 0\).</li>
                <li>Nếu tiệm cận đứng nằm <strong>bên trái</strong> Oy thì \(-\dfrac{d}{c} < 0 \Rightarrow \dfrac{d}{c} > 0\).</li>
                <li>Nếu tiệm cận đứng <strong>trùng với</strong> Oy thì \(-\dfrac{d}{c} = 0 \Rightarrow d = 0\).</li>
            </ul>
        </div>
        <img src="data//im/116.png" alt="minh họa">
    </section>

    <section>
        <h3>Tiệm cận ngang: \(y = \dfrac{a}{c}\)</h3>
        <div class="box procedure">
            <ul>
                <li>Nếu tiệm cận ngang nằm <strong>phía trên</strong> trục Ox thì \(\dfrac{a}{c} > 0\).</li>
                <li>Nếu tiệm cận ngang nằm <strong>phía dưới</strong> trục Ox thì \(\dfrac{a}{c} < 0\).</li>
                <li>Nếu tiệm cận ngang <strong>trùng với</strong> trục Ox thì \(a=0\).</li>
            </ul>
        </div>
        <img src="data//im/117.png" alt="minh họa">
    </section>

    <section>
        <h3>Tính đơn điệu hàm số</h3>
        <div class="box procedure">
            <ul>
                <li>Đạo hàm \(y' = \dfrac{ad-bc}{(cx+d)^2}\).</li>
                <li>Nếu mỗi nhánh đồ thị hàm số đi <strong>lên</strong> thì \(y' > 0, \forall x \ne -\dfrac{d}{c} \Rightarrow \boxed{ad-bc > 0}\).</li>
                <li>Nếu mỗi nhánh đồ thị hàm số đi <strong>xuống</strong> thì \(y' < 0, \forall x \ne -\dfrac{d}{c} \Rightarrow \boxed{ad-bc < 0}\).</li>
            </ul>
        </div>
        <img src="data//im/118.png" alt="minh họa">
    </section>

    <section>
        <h3>Giao điểm của đồ thị hàm số với trục Ox</h3>
        <div class="box procedure">
            <ul>
                <li>Giao điểm giữa đồ thị hàm số \(y = \dfrac{ax+b}{cx+d}\) với trục Ox là điểm \(M\left(-\dfrac{b}{a}; 0\right)\) với \(a \ne 0\).</li>
                <li>Nếu điểm M nằm <strong>bên phải gốc tọa độ O</strong> thì \(-\dfrac{b}{a} > 0 \Rightarrow \dfrac{b}{a} < 0\).</li>
                <li>Nếu điểm M nằm <strong>bên trái gốc tọa độ O</strong> thì \(-\dfrac{b}{a} < 0 \Rightarrow \dfrac{b}{a} > 0\).</li>
                <li>Nếu điểm M <strong>trùng với gốc tọa độ O</strong> thì \(-\dfrac{b}{a} = 0 \Rightarrow b = 0\).</li>
            </ul>
        </div>
        <img src="data//im/119.png" alt="minh họa">
    </section>

    <section>
        <h3>Giao điểm của đồ thị hàm số với trục Oy</h3>
        <div class="box procedure">
            <ul>
                <li>Giao điểm giữa đồ thị hàm số \(y = \dfrac{ax+b}{cx+d}\) với trục Oy là điểm \(N\left(0; \dfrac{b}{d}\right)\) với \(d \ne 0\).</li>
                <li>Nếu điểm N nằm <strong>phía trên gốc tọa độ O</strong> thì \(\dfrac{b}{d} > 0\).</li>
                <li>Nếu điểm N nằm <strong>phía dưới gốc tọa độ O</strong> thì \(\dfrac{b}{d} < 0\).</li>
                <li>Nếu điểm N <strong>trùng với gốc tọa độ O</strong> thì \(\dfrac{b}{d} = 0 \Rightarrow b = 0\).</li>
            </ul>
        </div>
    </section>

    <section>
        <section>
            <div class="box example">
                <h3>Ví dụ:</h3>
                <p>Đường cong trong hình vẽ bên là đồ thị hàm số nào dưới đây?</p>
                <ol>
                    <li>\(A.\) \(y = x^3 - 3x - 1\).</li>
                    <li>\(B.\) \(y = \dfrac{2x-1}{x-1}\).</li>
                    <li>\(C.\) \(y = \dfrac{x+1}{x-1}\).</li>
                    <li>\(D.\) \(y = x^4 + x^2 + 1\).</li>
                </ol>
                <img src="data//im/120.png" alt="minh họa">
            </div>
        </section>
        <section>
            <div class="box solution">
                <h4>📝 Lời giải</h4>
                <p><strong>Chọn C</strong></p>
                <p>Đồ thị đã cho là của hàm số nhất biến (bậc một trên bậc một) nên ta <strong>loại A, D</strong>.</p>
                <p>Tiệm cận đứng của đồ thị là \(x=1\), tiệm cận ngang của đồ thị là \(y=1\). <strong>Loại B</strong>.</p>
            </div>
        </section>
    </section>

    <section>
        <section>
            <div class="box example">
                <h3>Ví dụ:</h3>
                <p>Hình vẽ bên là đồ thị của hàm số nào sau đây?</p>
                <ol>
                    <li>\(A.\) \(y = \dfrac{2x+3}{x+1}\).</li>
                    <li>\(B.\) \(y = \dfrac{2x+1}{x-1}\).</li>
                    <li>\(C.\) \(y = \dfrac{2x-1}{x+1}\).</li>
                    <li>\(D.\) \(y = \dfrac{-2x+1}{x+1}\).</li>
                </ol>
                <img src="data//im/121.bmp" alt="minh họa">
            </div>
        </section>
        <section>
            <div class="box solution">
                <h4>📝 Lời giải</h4>
                <p><strong>Chọn C</strong></p>
                <p>Tiệm cận đứng của đồ thị là \(x = -1\), tiệm cận ngang của đồ thị là \(y=2\). <strong>Loại B, D</strong>.</p>
                <p>Ta thấy mỗi nhánh của đồ thị hàm số đã cho đi lên (từ trái sang phải) nên \(y' > 0\).</p>
                <p>Xét A: \(y = \dfrac{2x+3}{x+1} \Rightarrow y' = \dfrac{-1}{(x+1)^2} < 0, \forall x \ne -1\) (loại).</p>
                <p>Xét C: \(y = \dfrac{2x-1}{x+1} \Rightarrow y' = \dfrac{3}{(x+1)^2} > 0, \forall x \ne -1\) (thỏa mãn).</p>
            </div>
        </section>
    </section>

    <section>
        <section>
            <div class="box example">
                <h3>Ví dụ:</h3>
                <p>Cho hàm số \(y = \dfrac{ax+b}{cx+d}\) có đồ thị như trong hình bên dưới. Biết rằng \(a\) là số thực dương, hỏi trong các số \(b, c, d\) có tất cả bao nhiêu số dương?</p>
                <ol>
                    <li>\(A.\) 1.</li>
                    <li>\(B.\) 2.</li>
                    <li>\(C.\) 0.</li>
                    <li>\(D.\) 3.</li>
                </ol>
                <img src="data//im/122.png" alt="minh họa">
            </div>
        </section>
        <section>
            <div class="box solution">
                <h4>📝 Lời giải</h4>
                <p><strong>Chọn B</strong></p>
                <p>Tiệm cận ngang của đồ thị nằm phía trên \(Ox\) nên \(y = \dfrac{a}{c} > 0\) mà \(a>0 \Rightarrow c>0\).</p>
                <p>Tiệm cận đứng của đồ thị nằm bên trái \(Oy\) nên \(x = -\dfrac{d}{c} < 0 \Rightarrow \dfrac{d}{c} > 0\) mà \(c>0 \Rightarrow d>0\).</p>
                <p>Giao điểm của đồ thị hàm số với \(Oy\) là \(\left(0; \dfrac{b}{d}\right)\) nằm dưới \(O\) nên \(\dfrac{b}{d} < 0\) mà \(d>0 \Rightarrow b<0\).</p>
                <p>Vậy \(b < 0, c > 0, d > 0\).</p>
            </div>
        </section>
    </section>

    <section>
        <section>
            <div class="box example">
                <h3>Ví dụ:</h3>
                <p>Cho hàm số \(y = \dfrac{ax-b}{x-1}\) có đồ thị như hình vẽ dưới đây: Khẳng định nào sau đây đúng?</p>
                <ol>
                    <li>\(A.\) \(b < a < 0\).</li>
                    <li>\(B.\) \(a < b < 0\).</li>
                    <li>\(C.\) \(b > a\) và \(a < 0\).</li>
                    <li>\(D.\) \(a < 0 < b\).</li>
                </ol>
                <img src="data//im/123.png" alt="minh họa">
            </div>
        </section>
        <section>
            <div class="box solution">
                <h4>📝 Lời giải</h4>
                <p><strong>Chọn A</strong></p>
                <p>Tiệm cận ngang của đồ thị hàm số là \(y = \dfrac{a}{1} = -1 \Rightarrow a = -1\).</p>
                <p>Giao điểm của đồ thị hàm số với \(Oy\) là \((0; b) = (0; -2) \Rightarrow b = -2\).</p>
                <p>Vậy \(b < a < 0\).</p>
            </div>
        </section>
    </section>

    <section>
        <section>
            <div class="box example">
                <h3>Ví dụ:</h3>
                <p>Cho hàm số \(y = \dfrac{ax+b}{cx+d}\) có đồ thị như hình vẽ bên. Mệnh đề nào dưới đây đúng?</p>
                <ol>
                    <li>\(A.\) \(ac > 0, bd > 0\).</li>
                    <li>\(B.\) \(ab < 0, cd < 0\).</li>
                    <li>\(C.\) \(bc > 0, ad < 0\).</li>
                    <li>\(D.\) \(bc < 0, ad > 0\).</li>
                </ol>
                <img src="data//im/124.png" alt="minh họa">
            </div>
        </section>
        <section>
            <div class="box solution">
                <h4>📝 Lời giải</h4>
                <p><strong>Chọn C</strong></p>
                <p>Tiệm cận đứng đồ thị nằm bên phải \(Oy\) nên \(-\dfrac{d}{c} > 0 \Rightarrow cd < 0\) (1).</p>
                <p>Tiệm cận ngang của đồ thị nằm trên \(Ox\) nên \(\dfrac{a}{c} > 0 \Rightarrow ac > 0\) (2).</p>
                <p>Lấy (1) chia (2) theo vế: \(\dfrac{cd}{ac} < 0 \Rightarrow \dfrac{d}{a} < 0 \Rightarrow ad < 0\).</p>
                <p>Giao điểm của đồ thị với \(Oy\) là \(\left(0; \dfrac{b}{d}\right)\) nằm dưới điểm \(O\) nên \(\dfrac{b}{d} < 0 \Rightarrow bd < 0\) (3).</p>
                <p>Lấy (1) chia (3) theo vế: \(\dfrac{cd}{bd} > 0 \Rightarrow \dfrac{c}{b} > 0 \Rightarrow bc > 0\).</p>
            </div>
        </section>
    </section>
</section>

<section>
    <h2>Nhận diện đồ thị hàm phân thức bậc 2</h2>
    <section>
        <h3>Hàm số dạng \(y = \dfrac{ax^2+bx+c}{mx+n}\) với \(am \ne 0\)</h3>
        <div class="box procedure">
            <ul>
                <li>Tập xác định: \(D = \mathbb{R} \setminus \left\{-\dfrac{n}{m}\right\}\) và đạo hàm \(y' = \dfrac{am \cdot x^2 + 2an \cdot x + bn - mc}{(mx+n)^2}\).</li>
                <li>Hướng bên phải của đồ thị đi lên khi a, m cùng dấu và đi xuống khi a, m trái dấu.</li>
            </ul>
        </div>
        <img src="data//im/127.png" alt="minh họa">
    </section>
    <section>
        <div class="box note">
            <h3>Lưu ý</h3>
            <p>Đồ thị hàm số \(y = \dfrac{ax^2+bx+c}{mx+n}\) (\(am \ne 0\))</p>
            <ul>
                <li>Có <strong>tiệm cận đứng</strong> là đường thẳng \(x = -\dfrac{n}{m}\) và <strong>tiệm cận xiên</strong> là đường thẳng \(y = \dfrac{a}{m}x - \dfrac{an-bm}{m^2}\).</li>
                <li>Nhận giao điểm của tiệm cận đứng và tiệm cận xiên làm <strong>tâm đối xứng</strong>, tâm đối xứng này cũng là <strong>trung điểm</strong> của đoạn thẳng nối 2 điểm cực trị của đồ thị (nếu có).</li>
                <li>Nhận hai đường phân giác của các góc tạo bởi tiệm cận đứng và tiệm cận xiên làm <strong>trục đối xứng</strong>.</li>
                <li>Đường thẳng đi qua hai điểm cực trị (nếu có) có phương trình
                    \(y = \dfrac{(ax^2+bx+c)'}{(mx+n)'} = \dfrac{2ax+b}{m}\).</li>
            </ul>
        </div>
    </section>

    <section>
        <div class="box example">
            <h3>Ví dụ:</h3>
            <p>Bảng biến thiên bên dưới là của hàm số nào sau đây?</p>
            <img src="data//im/125.png" alt="minh họa">
            <ol>
                <li>\(A.\) \(y = \dfrac{2x^2+x+15}{-x-1}\).</li>
                <li>\(B.\) \(y = \dfrac{-x^2+x-18}{x+1}\).</li>
                <li>\(C.\) \(y = \dfrac{x^2+5x-24}{x+1}\).</li>
                <li>\(D.\) \(y = \dfrac{3x^2+3x+12}{-x-1}\).</li>
            </ol>
        </div>
    </section>

    <section>
        <div class="box example">
            <h3>Ví dụ:</h3>
            <p>Câu 87. Đồ thị hình bên là của hàm số nào sau đây ?</p>
            <ol>
                <li>\(A.\) \(y = \dfrac{2x^2+3x+1}{x+1}\).</li>
                <li>\(B.\) \(y = \dfrac{x^2+x+4}{x+1}\).</li>
                <li>\(C.\) \(y = \dfrac{-x^2-3x+10}{x+1}\).</li>
                <li>\(D.\) \(y = \dfrac{3x^2+5x-2}{x+1}\).</li>
            </ol>
            <img src="data//im/126.png" alt="minh họa">
        </div>
    </section>
</section>